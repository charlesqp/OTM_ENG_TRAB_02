% Experimentos e Resultados

Nesse momento executa-se os algoritmos utilizados para obtenção dos resultados que solucionam os problemas de programação não linear compostos por funções objetivos quadráticas e não quadráticas perfeitas. Diversos estudos da aplicação dos métodos Quase-Newton para a solução desses problemas foram realizados, entre os quais se destacam as análises relacionadas ao esforço computacional dos algoritmos e precisão. O primeiro envolve o número de iterações, o número de avaliações da função objetivo e o tempo de processamento. Já o segundo envolve o erro percentual da solução encontrada e do valor de função objetivo dessa solução. A seguir são apresentados os experimentos e resultados.

\section{1° Experimento: Avaliação das Funções Quadráticas}
    % Experimentos e Resultados

\pgfplotstableread{data/sect3s1/sec3s1tbl2.txt}{\tbla}
\pgfplotstableread{data/sect3s1/sec3s1tbl3.txt}{\tblb}
\pgfplotstableread{data/sect3s1/sec3s1tbl4.txt}{\tblc}
\pgfplotstableread{data/sect3s1/sec3s1tbl5.txt}{\tbld}

\pgfplotstableread{data/sect3s1/sect3s1f1_TEnd_Medio.txt}{\tblanovaT}
\pgfplotstableread{data/sect3s1/sect3s1f1_N_Iter_Medio.txt}{\tblanovaNI}
\pgfplotstableread{data/sect3s1/sect3s1f1_N_Evals_F_Medio.txt}{\tblanovaNE}
\pgfplotstableread{data/sect3s1/sect3s1f1_Erro_Xsol_P_Medio.txt}{\tblanovaEX}
\pgfplotstableread{data/sect3s1/sect3s1f1_Erro_F_P_Medio.txt}{\tblanovaEF}

\subsection{Primeira Função Analisada}

A primeira função analisada consiste na função \ref{eq:prifuntst} da seção \ref{sec:prifun}, reapresentada a seguir:

\begin{equation*} 
    f(x)= \frac{1}{2}*(x-c)^{T}*A*(x-c)
\end{equation*}

Para o experimento em questão a função objetivo é avaliada em duas dimensões. Nesse caso, o número máximo de iterações, calculado por meio da equação \ref{eq:maxevals}, é igual a $400$ e os limites inferior e superior das variáveis $[\-10, 10 ]\ $.

Na execução dos algoritmos, os métodos Quase-Newton, deverão considerados quatro pontos iniciais. Em dois pontos iniciais empregando a técnica da seção áurea diretamente na avaliação da função e, nos outros dois pontos, usando  a técnica da seção áurea feita por aproximações quadráticas para a função, conforme exibido abaixo:

\begin{itemize}
    \item $x_{0_1}=[\ 9 , 9 ]\ \rightarrow$ 1º Quadrante $\Rightarrow$ Aproximações Quadráticas  
    \item $x_{0_2}=[\ -3 , 2 ]\ \rightarrow$ 2º Quadrante $\Rightarrow$ Avaliação Direta da Função
    \item $x_{0_3}=[\ -8 , -6 ]\ \rightarrow$ 3º Quadrante $\Rightarrow$ Aproximações Quadráticas  
    \item $x_{0_4}=[\ 5 , -7 ]\ \rightarrow$ 4º Quadrante $\Rightarrow$ Avaliação Direta da Função
\end{itemize} 

Os resultados obtidos são apresentados nas tabelas abaixo.
\vspace{2mm}

    \begin{minipage}{\linewidth}
        \centering
        $x_{0_{1}}=[9,9]\Longrightarrow$  Aproximações Quadráticas para $f(x)$
        \label{tab:tbla} 
        \writetable{\tbla}
        \bigskip
        \captionof{table}{Resultados relacionados ao esforço computacional e precisão considerando o ponto inicial $x_{0_{1}}$}
    \end{minipage}

    \begin{minipage}{\linewidth}
        \centering
        $x_{0_{2}}=[-3,2]\Longrightarrow$  Avaliação Direta de $f(x)$
        \label{tab:tblb} 
        \writetable{\tblb}
        \bigskip
        \captionof{table}{Resultados relacionados ao esforço computacional e precisão considerando o ponto inicial $x_{0_{2}}$}
    \end{minipage}

    \begin{minipage}{\linewidth}
        \centering
        $x_{0_{3}}=[-8,-6]\Longrightarrow$  Aproximações Quadráticas para $f(x)$
        \label{tab:tblc} 
        \writetable{\tblc}
        \bigskip
        \captionof{table}{Resultados relacionados ao esforço computacional e precisão considerando o ponto inicial $x_{0_{3}}$}
    \end{minipage}
    
    \begin{minipage}{\linewidth}
        \centering
        $x_{0_{4}}=[5,-7]\Longrightarrow$  Avaliação Direta de $f(x)$
        \label{tab:tbld} 
        \writetable{\tbld}
        \bigskip
        \captionof{table}{Resultados relacionados ao esforço computacional e precisão considerando o ponto inicial $x_{0_{4}}$}
    \end{minipage}

\subsubsection{Análise Anova}

    \begin{minipage}[h!]{\linewidth}
        \centering
        \hrule
        \vspace{2mm}
        {Tabela Anova de 2 Fatores com blocos Aleatorizados}
        \vspace{2mm}
        \noindent
        \hrule 
        \vspace{2mm}
        Tempo de Processamento Médio\\
        \label{tab:tblDa} 
        \writeanova{\tblanovaT}\par
        \bigskip
        \centering
        Número de Iterações\\
        \label{tab:tblDb} 
        \writeanova{\tblanovaNI}\par
        \bigskip
        \centering
        Número de Avaliações de $f(x)$\\
        \label{tab:tblDc} 
        \writeanova{\tblanovaNE}\par
        \bigskip
        \centering
        {Erro de $X_{sol}(\%)$}\\
        \label{tab:tblDb} 
        \writeanova{\tblanovaEX}\par
        \bigskip
        \centering
        {Erro de $F(X_{sol})(\%)$}\\
        \label{tab:tblDb} 
        \writeanova{\tblanovaEF}\par
        \vspace{2mm}
        \hrule
        \vspace{2mm}
        \captionof{table}{Análise de variância dos Métodos Quase-Newton e da técnica da seção áurea para a primeira função analisada do primeiro experimento}
    \end{minipage}

    Para funções quadráticas, os métodos Quase-Newton:
    \begin{itemize}
        \item {Tempo (s):} O teste F da tabela anova indica que o método de cálculo da Seção Áurea interfere significativamente para a variação dos tempos médios de execução. A observação dos resultados aponta que as aproximações Quadráticas para $F(s)$ apresentam um tempo médio de execução 10 vezes menor. Da mesma maneira, o teste F da tabela anova indicou que o método otimização não interfere significativamente para a variação dos tempos médios de execução. 
        \item {Iterações:} Não variou significativamente para qualquer uma das condições. O teste F da tabela anova indica que o método de cálculo da Seção Áurea interfere significativamente para a variação do número de Iterações. Aparentemente, pela observação direta dos resultados, tem-se uma variação devida ao quadrante do ponto inicial que foi cancelada.
        \item {Avaliações de$F(x)$:} O teste F da tabela anova indica que o método de cálculo da Seção Áurea interfere significativamente para a variação do Número de avaliações de $F(x)$. A observação dos resultados aponta que as aproximações Quadráticas para $F(s)$ apresentam um tempo médio de execução 40 vezes menor.
        \item {Erro $X_{sol}$ (\%):} O erro é muito pequeno para todas as todas as variações, $\leq 10^{-9}$. Aparentemente o quadrante do ponto inicial aumenta em $10^5$ o tempo erro de $X_{sol}$. Pontos iniciais nos $1\degree$ e $4\degree$ quadrantes tem erros maiores. Porém ainda são virtualmente zero. O teste F da tabela anova considera que nenhum dos fatores é capaz de causar alteração significativa do Erro $X_{sol}$.
        \item {Erro $F(x_{sol})$ (\%):} \textbf{Sempre} zeram o erro da Função objetivo calculado no ponto da solução para todas variações de todos os fatores. O teste F da tabela anova considera que nenhum dos fatores é capaz de causar alteração significativa do Erro $F(x_{sol})$ (\%), devido a esta variável ter valor constante e igual a 0 para todos os blocos executados.
    \end{itemize}
        
    % Experimentos e Resultados

\pgfplotstableread{data/sect3s1/sec3s1tbl6.txt}{\tble}
\pgfplotstableread{data/sect3s1/sec3s1tbl7.txt}{\tblf}

\pgfplotstableread{data/sect3s1/sect3s1f2_TEnd_Medio.txt}{\tblanovaT}
\pgfplotstableread{data/sect3s1/sect3s1f2_N_Iter_Medio.txt}{\tblanovaNI}
\pgfplotstableread{data/sect3s1/sect3s1f2_N_Evals_F_Medio.txt}{\tblanovaNE}
\pgfplotstableread{data/sect3s1/sect3s1f2_Erro_Xsol_P_Medio.txt}{\tblanovaEX}
\pgfplotstableread{data/sect3s1/sect3s1f2_Erro_F_P_Medio.txt}{\tblanovaEF}

\subsection{Segunda Função Analisada}

A Segunda função analisada consiste na função apresentada na seção \ref{sec:secfun}.

A função objetivo apresentada na equação (\ref{eq:secfuntst}) possui duas variáveis de decisão. Nesse caso, o número máximo de iterações, calculado por meio da equação \ref{eq:maxevals}, é igual a 400 e os limites das variáveis de decisão $[\ -10 , 10 ]\ $. No intuito de obter uma função quadrática, o parâmetro $a$ é igual a zero, resultando em:

\begin{equation*} 
    f(x) = 12*x_1^2 - 4*x_2^2 - 12*x_1x_2 + 2*x_1
\end{equation*}

Na execução dos algoritmos que implementam os métodos Quase-Newton deverão ser considerados dois pontos iniciais definidos aleatoriamente pelo aluno. Em um desses dois pontos iniciais será usada a técnica da seção áurea da avaliação direta da função e, no outro, por meio das aproximações quadráticas para a função a cada iteração. 

\begin{minipage}{\linewidth}
    \centering
    $x_0=[x_1,x_2]\Longrightarrow$  Avaliação Direta de $f(x)$
    \label{tab:tble} 
    \writetable{\tble}
    \bigskip
    \captionof{table}{Resultados relacionados ao esforço computacional e precisão considerando o ponto inicial $x_{0_{1}}$}
\end{minipage}

\begin{minipage}{\linewidth}
    \centering
    $x_0=[x_1,x_2]\Longrightarrow$  Aproximações Quadráticas para $f(x)$
    \label{tab:tblf} 
    \writetable{\tblf}
    \bigskip
    \captionof{table}{Resultados relacionados ao esforço computacional e precisão considerando o ponto inicial $x_{0_{2}}$}
\end{minipage}  
   
\subsubsection{Análise Anova}

    \begin{minipage}[h!]{\linewidth}
        \centering
        \hrule
        \vspace{2mm}
        {Tabela Anova de 2 Fatores com blocos Aleatorizados}
        \vspace{2mm}
        \noindent
        \hrule 
        \vspace{2mm}
        Tempo de Processamento Médio\\
        \label{tab:tblDa} 
        \writeanova{\tblanovaT}\par
        \bigskip
        \centering
        Número de Iterações\\
        \label{tab:tblDb} 
        \writeanova{\tblanovaNI}\par
        \bigskip
        \centering
        Número de Avaliações de $f(x)$\\
        \label{tab:tblDc} 
        \writeanova{\tblanovaNE}\par
        \bigskip
        \centering
        {Erro de $X_{sol}(\%)$}\\
        \label{tab:tblDb} 
        \writeanova{\tblanovaEX}\par
        \bigskip
        \centering
        {Erro de $F(X_{sol})(\%)$}\\
        \label{tab:tblDb} 
        \writeanova{\tblanovaEF}\par
        \vspace{2mm}
        \hrule
        \vspace{2mm}
        \captionof{table}{Análise de variância dos Métodos Quase-Newton e da técnica da seção áurea para a segunda função analisada do primeiro experimento}
    \end{minipage}

    Para funções quadráticas, os métodos Quase-Newton:
    \begin{itemize}
        \item {Tempo (s):} O teste F da tabela anova indica que, tanto o método de cálculo da Seção Áurea quanto o método de otimização, interferem significativamente para a variação dos tempos médios de execução. Sendo a influência do método de cálculo da Seção Áurea é mais significativa. A observação dos resultados aponta que as aproximações Quadráticas para $F(s)$ apresentam um tempo médio de execução 10 vezes menor. Enquanto isso, o método otimização interfere com uma variância no tempo de execução significativamente menor, quase desprezível se comparada a variação causada pelo método de cálculo da Seção Áurea. 
        \item {Iterações:} O teste F da tabela anova considera que nenhum dos fatores é capaz de causar alteração significativa do número de iterações do algoritmo, devido a esta variável ter valor constante e igual a 6 para todos os blocos executados.
        \item {Avaliações de $F(x)$:} O teste F da tabela anova indica que o método de cálculo da Seção Áurea interfere significativamente para a variação do Número de avaliações de $F(x)$. A observação dos resultados aponta que as aproximações Quadráticas para $F(s)$ apresentam um tempo médio de execução 50 vezes menor. 
        \item {Erro $X_{sol}$ (\%):} O teste F da tabela anova indica que o Método de otimização interfere significativamente para a variação do Erro $X_{sol}$ (\%). Pela observação dos dados pode-se constatar que a combinação entre o uso do médoto de Otimização DFP provê o menor valor para o Erro $X_{sol}$ com uma redução da ordem de 10 vezes o valor das demais combinações.
        \item {Erro $F(x_{sol})$ (\%):} \textbf{Sempre} zeram o erro da Função objetivo calculado no ponto da solução para todas variações de todos os fatores. O teste F da tabela anova considera que nenhum dos fatores é capaz de causar alteração significativa do Erro $F(x_{sol})$ (\%), devido a esta variável ter valor constante e igual a 0 para todos os blocos executados.
    \end{itemize}    
    % Experimentos e Resultados

\pgfplotstableread{data/sect3s1/sec3s1tbl8.txt}{\tblg}
\pgfplotstableread{data/sect3s1/sec3s1tbl9.txt}{\tblh}

\pgfplotstableread{data/sect3s1/sect3s1f3_TEnd_Medio.txt}{\tblanovaT}
\pgfplotstableread{data/sect3s1/sect3s1f3_N_Iter_Medio.txt}{\tblanovaNI}
\pgfplotstableread{data/sect3s1/sect3s1f3_N_Evals_F_Medio.txt}{\tblanovaNE}
\pgfplotstableread{data/sect3s1/sect3s1f3_Erro_Xsol_P_Medio.txt}{\tblanovaEX}
\pgfplotstableread{data/sect3s1/sect3s1f3_Erro_F_P_Medio.txt}{\tblanovaEF}

\subsection{Terceira Função Analisada}

A terceira função analisada consiste na função da seção \ref{sec:terfun}.

A função objetivo apresentada acima possui duas variáveis de decisão. Nesse caso, o número máximo de iterações, calculado por meio da equação \ref{eq:maxevals}, é igual a 400 e os limites das variáveis de decisão definem o seguinte intervalo $[\ 0 , 3 ]\ $. Nesse intervalo, embora a função $f(x)$ possua dois termos cúbicos, a mesma se comporta como uma função quadrática. Sabendo que $a=\sqrt{2}$, tem-se que:

\begin{equation*} 
    f(x) = -8*x_1*x_2+2*x_2*x_1^3+2*x_1*x_2^3
\end{equation*}

Na execução dos algoritmos que implementam os métodos Quase-Newton, na análise dessa função, serão considerados dois pontos iniciais definidos aleatoriamente pelo aluno. Em um desses dois pontos iniciais será utilizada a técnica da seção áurea feita através da avaliação direta da função e, no outro, por meio das aproximações quadráticas para a função a cada iteração.

\begin{minipage}{\linewidth}
    \centering
    $x_0=[x_1,x_2]\Longrightarrow$  Avaliação Direta de $f(x)$
    \label{tab:tblg} 
    \writetable{\tblg}
    \bigskip
    \captionof{table}{Resultados relacionados ao esforço computacional e precisão considerando o ponto inicial $x_{0_{1}}$}
\end{minipage}

\begin{minipage}{\linewidth}
    \centering
    $x_0=[x_1,x_2]\Longrightarrow$  Aproximações Quadráticas para $f(x)$
    \label{tab:tblh} 
    \writetable{\tblh}
    \bigskip
    \captionof{table}{Resultados relacionados ao esforço computacional e precisão considerando o ponto inicial $x_{0_{2}}$}
\end{minipage}

\subsubsection{Análise Anova}

    \begin{minipage}[h!]{\linewidth}
    \centering
    \hrule
    \vspace{2mm}
    {Tabela Anova de 2 Fatores com blocos Aleatorizados}
    \vspace{2mm}
    \noindent
    \hrule 
    \vspace{2mm}
    Tempo de Processamento Médio\\
    \label{tab:tblDa} 
    \writeanova{\tblanovaT}\par
    \bigskip
    \centering
    Número de Iterações\\
    \label{tab:tblDb} 
    \writeanova{\tblanovaNI}\par
    \bigskip
    \centering
    Número de Avaliações de $f(x)$\\
    \label{tab:tblDc} 
    \writeanova{\tblanovaNE}\par
    \bigskip
    \centering
    {Erro de $X_{sol}(\%)$}\\
    \label{tab:tblDb} 
    \writeanova{\tblanovaEX}\par
    \bigskip
    \centering
    {Erro de $F(X_{sol})(\%)$}\\
    \label{tab:tblDb} 
    \writeanova{\tblanovaEF}\par
    \vspace{2mm}
    \hrule
    \vspace{2mm}
    \captionof{table}{Análise de variância dos Métodos Quase-Newton e da técnica da seção áurea para a terceira função analisada do primeiro experimento}
    \end{minipage}
    
    Para funções quadráticas, os métodos Quase-Newton:
    \begin{itemize}
        \item {Tempo (s):} {Tempo (s):} O teste F da tabela anova indica que, tanto o método de cálculo da Seção Áurea quanto o método de otimização, interferem significativamente para a variação dos tempos médios de execução. Sendo que a influência do método de cálculo da Seção Áurea é mais significativa. A observação dos resultados aponta que as aproximações Quadráticas para $F(s)$ apresentam um tempo médio de execução 100 vezes maior. Enquanto isso, o método otimização interfere com uma variância no tempo de execução significativamente menor, demonstrando que o método de otimização de Huang obtem um tempo de execução que é a metade dos tempos dos outros métodos em para qualquer um dos métodos de cálculo da seção áurea.
        \item {Iterações:}  O teste F da tabela anova indica que, tanto o método de cálculo da Seção Áurea quanto o método de otimização, interferem significativamente para a variação do número de iterações do algoritmo de otimização. Sendo que a influência do método de cálculo da Seção Áurea é o fator mais significativo. A observação dos resultados aponta que as aproximações Quadráticas para $F(s)$ apresentam um tempo médio de execução 5 vezes maior, onde a combinação crítica com o pior resultado foi empregar o método de otimização de Huang com aproximações quadráticas para o cálculo de $f(X)$, com 399 iterações. Por outro lado, o melhor resultado para o número de iterações é empregar o método de Huang com avaliação direta da função $f(X)$ para o cálculo da seção áurea, com 11 iterações.
        \item {Avaliações de $F(x)$:} O teste F da tabela anova indica que, tanto o método de cálculo da Seção Áurea quanto o método de otimização, interferem significativamente para a variação do número de Avaliações de $F(x)$. Sendo que a influência do método de cálculo da Seção Áurea é o fator mais significativo. A observação dos resultados aponta que as aproximações quadráticas para $F(s)$ apresentam um número de Avaliações de $F(x)$ que é, em média, 180 vezes maior do que quando comparado a avaliação direta de $F(s)$ no cálculo da seção áurea. Assim como aconteceu com o número de iterações. O pior resultado foi a combinação entre o método de otimização de Huang com aproximações quadráticas para $F(s)$, com 17701 avaliações de $F(x)$, enquanto que a melhor combinação foi o método de otimização de Huang com avaliação direta de $F(s)$.        
        \item {Erro $X_{sol}$ (\%):} O teste F da tabela anova indica que, tanto o método de cálculo da Seção Áurea quanto o método de otimização, interferem significativamente para a variação do Erro $X_{sol}$ (\%). Sendo que a influência do método de cálculo da Seção Áurea é o fator mais significativo. A observação dos resultados aponta que as aproximações quadráticas para $F(s)$ apresentam um Erro $X_{sol}$ (\%) que é, em média, $10^(10)$ vezes maior do que quando se empregam avaliações diretas de $F(x)$ para o cálculo da seção áurea. O pior resultado foi a combinação entre o método de otimização de Huang com aproximações quadráticas para $F(s)$, com Erro $X_{sol}$ de 964.19 (\%), enquanto que a melhor combinação foi o método de otimização de DFP com avaliação direta de $F(s)$ com Erro $X_{sol}$ de $2.73*10^{-7}$ (\%).  
        \item {Erro $F(x_{sol})$ (\%):} O teste F da tabela anova indica que, tanto o método de cálculo da Seção Áurea quanto o método de otimização, interferem significativamente para a variação do Erro $F(x_{sol})$ (\%). Sendo que a influência do método de cálculo da Seção Áurea é mais significativa. A observação dos resultados aponta que as aproximações Quadráticas para $F(s)$ apresentam um Erro $F(x_{sol})$ (\%) diferente de zero, em média da ordem de $10*{-10}$, sendo o pior caso a quando combinados com o método de Huang. Enquanto isso, a avaliação direta da função $F(x)$ para o cálculo da seção áurea sempre zera o Erro $F(x_{sol})$ (\%) para qualquer método de otimização empregado.
    \end{itemize}
   
\newpage
\section{2° Experimento: Avaliação das Funções Não Quadráticas}
    % Experimentos e Resultados

A função analisada consiste na função da seção \ref{sec:secfun}.

A função objetivo apresentada possui duas variáveis de decisão. Nesse caso, o número máximo de iterações, calculado por meio da equação \ref{eq:maxevals}, é igual a 400 e os limites das variáveis de decisão definem o seguinte intervalo $[\ -10 , 10 ]\ $. No intuito de obter uma função não quadrática perfeita, foram considerados três valores do parâmetro $a$, os quais são apresentados abaixo seguido da respectiva função objetivo:

\begin{equation*}   
    \begin{aligned}
        \alpha &=-0.0263 \rightarrow &f(x) = 12*x_1^2 - 4*x_2^2 - 12*x_1x_2 + 2*x_1 -0.0263*(x_1^3+x_2^3)\\
        \alpha &=+0.0263 \rightarrow &f(x) = 12*x_1^2 - 4*x_2^2 - 12*x_1x_2 + 2*x_1 + 0.0263*(x_1^3+x_2^3)\\
        \alpha &=1 \rightarrow &f(x) = 12*x_1^2 - 4*x_2^2 - 12*x_1x_2 + 2*x_1 + (x_1^3+x_2^3)\\   
    \end{aligned}
\end{equation*}

Para cada valor do parâmetro $a$ deverão ser considerados dois pontos iniciais, dentre os quais um será executado considerando a técnica da seção áurea feita através da avaliação direta da função e o outro considerando as aproximações quadráticas para a função a cada iteração.

Vale ressaltar que o parâmetro a pondera o termo cúbico da função objetivo $f(x)$ do problema de otimização estudado, quanto maior o valor absoluto desse parâmetro, maior é a influência do termo não quadrático na função. Para esta função os valores assumidos para parâmetro  a que não comprometem a convexidade da função   no intervalo $[\ -10 , 10 ]\ $ estão no intervalo: \\ {\centering $-0.0263 \leq a \leq 0.0263$} \\
\vspace{2mm}
    % Experimentos e Resultados

\pgfplotstableread{data/sect3s2/sec3s2tbl10a.txt}{\tbli}
\pgfplotstableread{data/sect3s2/sec3s2tbl10b.txt}{\tblj}

\pgfplotstableread{data/sect3s2/sect3s2f1_TEnd_Medio.txt}{\tblanovaT}
\pgfplotstableread{data/sect3s2/sect3s2f1_N_Iter_Medio.txt}{\tblanovaNI}
\pgfplotstableread{data/sect3s2/sect3s2f1_N_Evals_F_Medio.txt}{\tblanovaNE}
\pgfplotstableread{data/sect3s2/sect3s2f1_Erro_Xsol_P_Medio.txt}{\tblanovaEX}
\pgfplotstableread{data/sect3s2/sect3s2f1_Erro_F_P_Medio.txt}{\tblanovaEF}

\subsection{Primeira função analisada: $\alpha = -0.0263$}

\begin{minipage}[h!]{\linewidth}
    \centering
    Aproximações Quadráticas de $f(x)$
    \label{tab:tbli} 
    \writetable{\tbli}\par
    \bigskip
    \centering
    Avaliação Direta de $f(x)$
    \label{tab:tblj} 
    \writetable{\tblj}
    \captionof{table}{Resultados relacionados ao esforço computacional e precisão considerando $\alpha = -0.0263$}
\end{minipage}

\begin{minipage}[h!]{\linewidth}
    \centering
    \hrule
    \vspace{2mm}
    {Tabela Anova de 2 Fatores com blocos Aleatorizados}
    \vspace{2mm}
    \noindent
    \hrule 
    \vspace{2mm}
    Tempo de Processamento Médio\\
    \label{tab:tblDa} 
    \writeanova{\tblanovaT}\par
    \bigskip
    \centering
    Número de Iterações\\
    \label{tab:tblDb} 
    \writeanova{\tblanovaNI}\par
    \bigskip
    \centering
    Número de Avaliações de $f(x)$\\
    \label{tab:tblDc} 
    \writeanova{\tblanovaNE}\par
    \bigskip
    \centering
    {Erro de $X_{sol}(\%)$}\\
    \label{tab:tblDb} 
    \writeanova{\tblanovaEX}\par
    \bigskip
    \centering
    {Erro de $F(X_{sol})(\%)$}\\
    \label{tab:tblDb} 
    \writeanova{\tblanovaEF}\par
    \vspace{2mm}
    \hrule
    \vspace{2mm}
    \captionof{table}{Análise de variância dos Métodos Quase-Newton e da técnica da seção áurea para a primeira função analisada do primeiro experimento}
\end{minipage}

\subsubsection{Conclusão Parcial}
        Para funções quadráticas, os métodos Quase-Newton:
        \begin{itemize}
            \item {Tempo (s):} O teste F da tabela anova indica que ambos, o método ode otimização e o método de cálculo da Seção Áurea interferem significativamente para a variação dos tempos médios de execução. Sendo o método de cálculo da seção áurea o fator mais significativo. A observação dos resultados aponta que o melhor caso se obtém empregando as aproximações Quadráticas para $F(s)$ apresentam um tempo médio de execução 2 vezes menor. Entretanto a observação dos dados mostra também que apesar do desempenho médio empregando aproximações Quadráticas para $F(s)$ no cálculo da seção áurea, o menor tempo de execução se deu pela combinação do Método de otimização de BFGS e a avaliação direta de $F(s)$. Enquanto isso, o pior caso ocorreu quando combina-se o método de otimização de Huang e a avaliação direta de $F(s)$. 
            \item {Iterações:}  O teste F da tabela anova indica que ambos, o método de otimização e o método de cálculo da Seção Áurea interferem significativamente para a variação dos número de iterações do algoritmo. Sendo o método de cálculo da seção áurea o fator mais significativo. A observação dos resultados aponta que o melhor caso se dá pela combinação dos métodos DFP ou BFGS e a avaliação direta de $F(x)$. Da mesma maneira, o pior caso se dá pela combinação dos dos métodos Huang ou Biggs e Aproximações quadráticas de $F(x)$.
            \item {Avaliações de $F(x)$:} O teste F da tabela anova indica que ambos, o método de otimização e o método de cálculo da Seção Áurea interferem significativamente para a variação dos número de avaliações de $F(x)$. Sendo o método de cálculo da seção áurea o fator mais significativo. A observação dos resultados aponta que o melhor caso se dá pela combinação dos métodos DFP ou BFGS e aproximações quadráticas de $F(x)$. Da mesma maneira, o pior caso se dá pela combinação dos dos métodos Huang ou Biggs e a avaliação direta de $F(x)$.
            \item {Erro $X_{sol}$ (\%):} O teste F da tabela anova indica que ambos, o método de otimização e o método de cálculo da Seção Áurea interferem significativamente para a variação do erro $X_{sol}$ (\%). Sendo o método de cálculo da seção áurea o fator mais significativo. A observação dos resultados aponta que o melhor caso se dá pela combinação dos dos métodos DFP e aproximações quadráticas de $F(x)$. Da mesma maneira, o pior caso se dá pela combinação dos dos métodos DFP ou BFGS e a avaliação direta de $F(x)$. Pela observação dos dados os métodos interferem mais na variação do erro $X_{sol}$ (\%) de forma independente ao método de cálculo da seção áurea.
            \item {Erro $F(x_{sol})$ (\%):} O teste F da tabela anova indica que ambos, o método de otimização e o método de cálculo da Seção Áurea interferem significativamente para a variação do erro $F(x_{sol})$ (\%). Sendo o método de cálculo da seção áurea o fator mais significativo. A observação dos resultados aponta que o melhor caso se dá pela combinação dos dos métodos Huang ou Biggs e aproximações quadráticas de $F(x)$ que são capazes de zerar por completo o erro. Todas as outras combinações tem o mesmo efeito sobre o erro $F(x_{sol})$ (\%), tornando-o da ordem de $10^{-15}$.
        \end{itemize}
    % Experimentos e Resultados

\pgfplotstableread{data/sect3s2/sec3s2tbl11a.txt}{\tblk}
\pgfplotstableread{data/sect3s2/sec3s2tbl11b.txt}{\tbll}

\pgfplotstableread{data/sect3s2/sect3s2f2_TEnd_Medio.txt}{\tblanovaT}
\pgfplotstableread{data/sect3s2/sect3s2f2_N_Iter_Medio.txt}{\tblanovaNI}
\pgfplotstableread{data/sect3s2/sect3s2f2_N_Evals_F_Medio.txt}{\tblanovaNE}
\pgfplotstableread{data/sect3s2/sect3s2f2_Erro_Xsol_P_Medio.txt}{\tblanovaEX}
\pgfplotstableread{data/sect3s2/sect3s2f2_Erro_F_P_Medio.txt}{\tblanovaEF}

\subsection{Segunda função analisada: $\alpha = +0.0263$}

\begin{minipage}[h!]{\linewidth}
    \centering
    {Aproximações Quadráticas de $f(x)$}
    \label{tab:tblk} 
    \writetable{\tblk}\par
    \bigskip
    \centering
    {Avaliação Direta de $f(x)$}
    \label{tab:tbll} 
    \writetable{\tbll}
    \captionof{table}{Resultados relacionados ao esforço computacional e precisão considerando $\alpha = 0.0263$}
\end{minipage}

\begin{minipage}[h!]{\linewidth}
    \centering
    \hrule
    \vspace{2mm}
    {Tabela Anova de 2 Fatores com blocos Aleatorizados}
    \vspace{2mm}
    \noindent
    \hrule 
    \vspace{2mm}
    Tempo de Processamento Médio\\
    \label{tab:tblDa} 
    \writeanova{\tblanovaT}\par
    \bigskip
    \centering
    Número de Iterações\\
    \label{tab:tblDb} 
    \writeanova{\tblanovaNI}\par
    \bigskip
    \centering
    Número de Avaliações de $f(x)$\\
    \label{tab:tblDc} 
    \writeanova{\tblanovaNE}\par
    \bigskip
    \centering
    {Erro de $X_{sol}(\%)$}\\
    \label{tab:tblDb} 
    \writeanova{\tblanovaEX}\par
    \bigskip
    \centering
    {Erro de $F(X_{sol})(\%)$}\\
    \label{tab:tblDb} 
    \writeanova{\tblanovaEF}\par
    \vspace{2mm}
    \hrule
    \vspace{2mm}
    \captionof{table}{Análise de variância dos Métodos Quase-Newton e da técnica da seção áurea para a primeira função analisada do primeiro experimento}
\end{minipage}

\subsubsection{Conclusão Parcial}
        Para funções quadráticas, os métodos Quase-Newton:
        \begin{itemize}
            \item {Tempo (s):} O teste F da tabela anova indica que o método de cálculo da Seção Áurea interfere significativamente para a variação dos tempos médios de execução. A observação dos resultados aponta que a média dos tempos de execução é menor quando se empregam as aproximações Quadráticas para $F(s)$.
            \item {Iterações:} O teste F da tabela anova indica que ambos, o método de otimização e o método de cálculo da Seção Áurea interferem significativamente para a variação dos número de iterações do algoritmo. Sendo o método de cálculo da seção áurea o fator mais significativo. A observação dos resultados aponta que o número de iterações é menor quando se empregam as aproximações Quadráticas para $F(s)$. No melhor caso, os algoritmos DFP e BFGS performam melhor para qualquer uma das escolhas do método de cálculo da seção áurea. No pior caso, os métodos de Huang e Biggs tem a pior performance quando combinados com a avaliação direta de $F(x)$
            \item {Avaliações de $F(x)$:} O teste F da tabela anova indica que ambos, o método de otimização e o método de cálculo da Seção Áurea interferem significativamente para a variação dos número de Avaliações de $F(x)$. Sendo o método de cálculo da seção áurea o fator mais significativo. Para o melhor caso, o emprego de aproximações quadráticas de $F(x)$ obtém o menor valor para todas os métodos de otimização. Para o pior caso, a combinação do método de Huang com a avaliação direta de $F(x)$ obtém o maior valor. A relação entre o melhor e o pior é de 73 vezes.
            \item {Erro $X_{sol}$ (\%):} O teste F da tabela anova indica que ambos, o método de otimização e o método de cálculo da Seção Áurea interferem significativamente para a variação do erro $X_{sol}$. Sendo o método de cálculo da seção áurea o fator mais significativo. Para o melhor caso, o emprego de aproximações quadráticas de $F(x)$ combinado com o método de Huang obtém o menor valor. Para o pior caso, a combinação da avaliação direta de $F(x)$ com qualquer método de otimização obtém o maior valor.
            \item {Erro $F(x_{sol})$ (\%):} \textbf{Sempre} zeram o erro da Função objetivo calculado no ponto da solução para todas variações de todos os fatores, o valor obtido $10^{-16}$ é praticamente zero pois é da mesma ordem de valor da precisão relativa do ponto flutuante no Matlab. O teste F da tabela anova considera que nenhum dos fatores é capaz de causar alteração significativa do Erro $F(x_{sol})$ (\%), devido a esta variável ter valor constante e igual a 0 para todos os blocos executados.
        \end{itemize}  
    % Experimentos e Resultados

\pgfplotstableread{data/sect3s2/sec3s2tbl12a.txt}{\tblm}
\pgfplotstableread{data/sect3s2/sec3s2tbl12b.txt}{\tbln}

\pgfplotstableread{data/sect3s2/sect3s2f3_TEnd_Medio.txt}{\tblanovaT}
\pgfplotstableread{data/sect3s2/sect3s2f3_N_Iter_Medio.txt}{\tblanovaNI}
\pgfplotstableread{data/sect3s2/sect3s2f3_N_Evals_F_Medio.txt}{\tblanovaNE}
\pgfplotstableread{data/sect3s2/sect3s2f3_Erro_Xsol_P_Medio.txt}{\tblanovaEX}
\pgfplotstableread{data/sect3s2/sect3s2f3_Erro_F_P_Medio.txt}{\tblanovaEF}

\subsection{Terceira função analisada: $\alpha = 0$}

\begin{minipage}[h!]{\linewidth}
    \centering
    {Aproximações Quadráticas de $f(x)$}
    \label{tab:tblm} 
    \writetable{\tblm}\par
    \bigskip
    \centering
    {Avaliação Direta de $f(x)$}
    \label{tab:tbln} 
    \writetable{\tbln}     
    \captionof{table}{Resultados relacionados ao esforço computacional e precisão considerando $\alpha = 1.0$}
\end{minipage}

\begin{minipage}[h!]{\linewidth}
    \centering
    \hrule
    \vspace{2mm}
    {Tabela Anova de 2 Fatores com blocos Aleatorizados}
    \vspace{2mm}
    \noindent
    \hrule 
    \vspace{2mm}
    Tempo de Processamento Médio\\
    \label{tab:tblDa} 
    \writeanova{\tblanovaT}\par
    \bigskip
    \centering
    Número de Iterações\\
    \label{tab:tblDb} 
    \writeanova{\tblanovaNI}\par
    \bigskip
    \centering
    Número de Avaliações de $f(x)$\\
    \label{tab:tblDc} 
    \writeanova{\tblanovaNE}\par
    \bigskip
    \centering
    {Erro de $X_{sol}(\%)$}\\
    \label{tab:tblDb} 
    \writeanova{\tblanovaEX}\par
    \bigskip
    \centering
    {Erro de $F(X_{sol})(\%)$}\\
    \label{tab:tblDb} 
    \writeanova{\tblanovaEF}\par
    \vspace{2mm}
    \hrule
    \vspace{2mm}
    \captionof{table}{Análise de variância dos Métodos Quase-Newton e da técnica da seção áurea para a primeira função analisada do primeiro experimento}
\end{minipage}

\subsubsection{Conclusão Parcial}
        Para funções quadráticas, os métodos Quase-Newton:
        \begin{itemize}
        \item {Tempo (s):}  O teste F da tabela anova indica que o método de cálculo da Seção Áurea interfere significativamente para a variação dos tempos médios de execução. A observação dos resultados aponta que o número de iterações é menor quando se combina a avaliação direta de $F(s)$ com o método BFGS. No pior caso, o método de Huang tem a pior performance quando combinado com a Aproximação Quadrática de $F(x)$
        \item {Iterações:}   O teste F da tabela anova indica que o método de cálculo da Seção Áurea interfere significativamente para a variação do número de iterações. A observação dos resultados aponta que o número de iterações é menor quando se combina a avaliação direta de $F(s)$ com o método BFGS. No pior caso, os métodos de Huang ou Biggs tem a pior performance quando combinado com a Aproximação Quadrática de $F(x)$
        \item {Avaliações de $F(x)$:} O teste F da tabela anova indica que ambos, o método de cálculo da Seção Áurea e o método de otimização interferem significativamente para a variação dos número de Avaliações de $F(x)$. A observação dos resultados aponta que o número de avaliações de $F(x)$ é menor quando se combina a Aproximação Quadrática de $F(x)$ com o método DFP. No pior caso, o método DFP tem a pior performance quando combinado com a  avaliação direta de $F(s)$.
        \item {Erro $X_{sol}$ (\%):} O teste F da tabela anova indica que ambos, o método de cálculo da Seção Áurea e o método de otimização interferem significativamente para a variação do Erro $X_{sol}$ (\%). A observação dos resultados aponta que o Erro $X_{sol}$ (\%) é menor quando se combina a Avaliação Direta de $F(x)$ com o método DFP. No pior caso, o método de Huang tem a pior performance quando combinado com a Aproximação Quadrática $F(s)$.
        \item {Erro $F(x_{sol})$ (\%):}  O teste F da tabela anova indica que ambos, o método de cálculo da Seção Áurea e o método de otimização interferem significativamente para a variação do Erro $F(x_{sol})$ (\%). A observação dos resultados aponta que o Erro $X_{sol}$ (\%) é menor para a combinação da Avaliação Direta de $F(x)$ com qualquer método de otimização. No pior caso, o método de Huang tem a pior performance quando combinado com a Aproximação Quadrática $F(s)$.
        \end{itemize} 
\newpage   
\import{Parts/Part 3/}{section3-3}
\import{Parts/Part 3/}{section3-4}
\import{Parts/Part 3/}{section3-5}
\import{Parts/Part 3/}{section3-6}