% Experimentos e Resultados

\pgfplotstableread{data/sect3s1/sec3s1tbl2.txt}{\tbla}
\pgfplotstableread{data/sect3s1/sec3s1tbl3.txt}{\tblb}
\pgfplotstableread{data/sect3s1/sec3s1tbl4.txt}{\tblc}
\pgfplotstableread{data/sect3s1/sec3s1tbl5.txt}{\tbld}

\pgfplotstableread{data/sect3s1/sect3s1f1_TEnd_Medio.txt}{\tblanovaT}
\pgfplotstableread{data/sect3s1/sect3s1f1_N_Iter_Medio.txt}{\tblanovaNI}
\pgfplotstableread{data/sect3s1/sect3s1f1_N_Evals_F_Medio.txt}{\tblanovaNE}
\pgfplotstableread{data/sect3s1/sect3s1f1_Erro_Xsol_P_Medio.txt}{\tblanovaEX}
\pgfplotstableread{data/sect3s1/sect3s1f1_Erro_F_P_Medio.txt}{\tblanovaEF}

\subsection{Primeira Função Analisada}

A primeira função analisada consiste na função \ref{eq:prifuntst} da seção \ref{sec:prifun}, reapresentada a seguir:

\begin{equation*} 
    f(x)= \frac{1}{2}*(x-c)^{T}*A*(x-c)
\end{equation*}

Para o experimento em questão a função objetivo é avaliada em duas dimensões. Nesse caso, o número máximo de iterações, calculado por meio da equação \ref{eq:maxevals}, é igual a $400$ e os limites inferior e superior das variáveis $[\-10, 10 ]\ $.

Na execução dos algoritmos, os métodos Quase-Newton, deverão considerados quatro pontos iniciais. Em dois pontos iniciais empregando a técnica da seção áurea diretamente na avaliação da função e, nos outros dois pontos, usando  a técnica da seção áurea feita por aproximações quadráticas para a função, conforme exibido abaixo:

\begin{itemize}
    \item $x_{0_1}=[\ 9 , 9 ]\ \rightarrow$ 1º Quadrante $\Rightarrow$ Aproximações Quadráticas  
    \item $x_{0_2}=[\ -3 , 2 ]\ \rightarrow$ 2º Quadrante $\Rightarrow$ Avaliação Direta da Função
    \item $x_{0_3}=[\ -8 , -6 ]\ \rightarrow$ 3º Quadrante $\Rightarrow$ Aproximações Quadráticas  
    \item $x_{0_4}=[\ 5 , -7 ]\ \rightarrow$ 4º Quadrante $\Rightarrow$ Avaliação Direta da Função
\end{itemize} 

Os resultados obtidos são apresentados nas tabelas abaixo.
\vspace{2mm}

    \begin{minipage}{\linewidth}
        \centering
        $x_{0_{1}}=[9,9]\Longrightarrow$  Aproximações Quadráticas para $f(x)$
        \label{tab:tbla} 
        \writetable{\tbla}
        \bigskip
        \captionof{table}{Resultados relacionados ao esforço computacional e precisão considerando o ponto inicial $x_{0_{1}}$}
    \end{minipage}

    \begin{minipage}{\linewidth}
        \centering
        $x_{0_{2}}=[-3,2]\Longrightarrow$  Avaliação Direta de $f(x)$
        \label{tab:tblb} 
        \writetable{\tblb}
        \bigskip
        \captionof{table}{Resultados relacionados ao esforço computacional e precisão considerando o ponto inicial $x_{0_{2}}$}
    \end{minipage}

    \begin{minipage}{\linewidth}
        \centering
        $x_{0_{3}}=[-8,-6]\Longrightarrow$  Aproximações Quadráticas para $f(x)$
        \label{tab:tblc} 
        \writetable{\tblc}
        \bigskip
        \captionof{table}{Resultados relacionados ao esforço computacional e precisão considerando o ponto inicial $x_{0_{3}}$}
    \end{minipage}
    
    \begin{minipage}{\linewidth}
        \centering
        $x_{0_{4}}=[5,-7]\Longrightarrow$  Avaliação Direta de $f(x)$
        \label{tab:tbld} 
        \writetable{\tbld}
        \bigskip
        \captionof{table}{Resultados relacionados ao esforço computacional e precisão considerando o ponto inicial $x_{0_{4}}$}
    \end{minipage}

\subsubsection{Análise Anova}

    \begin{minipage}[h!]{\linewidth}
        \centering
        \hrule
        \vspace{2mm}
        {Tabela Anova de 2 Fatores com blocos Aleatorizados}
        \vspace{2mm}
        \noindent
        \hrule 
        \vspace{2mm}
        Tempo de Processamento Médio\\
        \label{tab:tblDa} 
        \writeanova{\tblanovaT}\par
        \bigskip
        \centering
        Número de Iterações\\
        \label{tab:tblDb} 
        \writeanova{\tblanovaNI}\par
        \bigskip
        \centering
        Número de Avaliações de $f(x)$\\
        \label{tab:tblDc} 
        \writeanova{\tblanovaNE}\par
        \bigskip
        \centering
        {Erro de $X_{sol}(\%)$}\\
        \label{tab:tblDb} 
        \writeanova{\tblanovaEX}\par
        \bigskip
        \centering
        {Erro de $F(X_{sol})(\%)$}\\
        \label{tab:tblDb} 
        \writeanova{\tblanovaEF}\par
        \vspace{2mm}
        \hrule
        \vspace{2mm}
        \captionof{table}{Análise de variância dos Métodos Quase-Newton e da técnica da seção áurea para a primeira função analisada do primeiro experimento}
    \end{minipage}

    Para funções quadráticas, os métodos Quase-Newton:
    \begin{itemize}
        \item {Tempo (s):} O teste F da tabela anova indica que o método de cálculo da Seção Áurea interfere significativamente para a variação dos tempos médios de execução. A observação dos resultados aponta que as aproximações Quadráticas para $F(s)$ apresentam um tempo médio de execução 10 vezes menor. Da mesma maneira, o teste F da tabela anova indicou que o método otimização não interfere significativamente para a variação dos tempos médios de execução. 
        \item {Iterações:} Não variou significativamente para qualquer uma das condições. O teste F da tabela anova indica que o método de cálculo da Seção Áurea interfere significativamente para a variação do número de Iterações. Aparentemente, pela observação direta dos resultados, tem-se uma variação devida ao quadrante do ponto inicial que foi cancelada.
        \item {Avaliações de$F(x)$:} O teste F da tabela anova indica que o método de cálculo da Seção Áurea interfere significativamente para a variação do Número de avaliações de $F(x)$. A observação dos resultados aponta que as aproximações Quadráticas para $F(s)$ apresentam um tempo médio de execução 40 vezes menor.
        \item {Erro $X_{sol}$ (\%):} O erro é muito pequeno para todas as todas as variações, $\leq 10^{-9}$. Aparentemente o quadrante do ponto inicial aumenta em $10^5$ o tempo erro de $X_{sol}$. Pontos iniciais nos $1\degree$ e $4\degree$ quadrantes tem erros maiores. Porém ainda são virtualmente zero. O teste F da tabela anova considera que nenhum dos fatores é capaz de causar alteração significativa do Erro $X_{sol}$.
        \item {Erro $F(x_{sol})$ (\%):} \textbf{Sempre} zeram o erro da Função objetivo calculado no ponto da solução para todas variações de todos os fatores. O teste F da tabela anova considera que nenhum dos fatores é capaz de causar alteração significativa do Erro $F(x_{sol})$ (\%), devido a esta variável ter valor constante e igual a 0 para todos os blocos executados.
    \end{itemize}
       