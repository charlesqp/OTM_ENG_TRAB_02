% Experimentos e Resultados

\pgfplotstableread{data/sect3s6s1/sec1s1tbl17_tEnd.txt}{\tblAa}
\pgfplotstableread{data/sect3s6s1/sec1s1tbl17_N_Iter.txt}{\tblAb}
\pgfplotstableread{data/sect3s6s1/sec1s1tbl17_N_Evals.txt}{\tblAc}
\pgfplotstableread{data/sect3s6s1/sec1s1tbl17_Erro_Xsol_P.txt}{\tblAd}
\pgfplotstableread{data/sect3s6s1/sec1s1tbl17_Erro_F_P.txt}{\tblAe}
\pgfplotstableread{data/sect3s6s1/sec1s1tbl17_DFP_raw.txt}{\tblADFP}
\pgfplotstableread{data/sect3s6s1/sec1s1tbl17_BFGS_raw.txt}{\tblABFGS}
\pgfplotstableread{data/sect3s6s1/sec1s1tbl17_Huang_raw.txt}{\tblAHuang}
\pgfplotstableread{data/sect3s6s1/sec1s1tbl17_Biggs_raw.txt}{\tblABiggs}

\begin{minipage}[h!]{\linewidth}            
    \centering
    \hrule
    \vspace{2mm}
    Técnica da Seção Áurea Feita por meio da avaliação direta de $F(x)$ \\ $\alpha=-0.0236$
    \vspace{2mm}
    \noindent
    \hrule 
    \vspace{2mm}
    Tempo de Processamento Médio\\
    \label{tab:tblAa} 
    \writetablestt{\tblAa}\par
    \bigskip
    \centering
    Número de Iterações\\
    \label{tab:tblAb} 
    \writetablestt{\tblAb}\par
    \bigskip
    \centering
    Número de Avaliações de $f(x)$\\
    \label{tab:tblAc} 
    \writetablestt{\tblAc}\par
    \vspace{2mm}
    \hrule
    \vspace{2mm}
    \captionof{table}{Análise estatística dos Métodos Quase-Newton para a técnica da seção áurea feita por meio da avaliação direta da função objetivo e parâmetro $\alpha=-0.0263$}
\end{minipage}

\pgfplotsset{width=7.5cm,height=6cm,compat=1.18}\usepgfplotslibrary{statistics}
\begin{figure}[h!]
    \centering            
    \subfloat[\centering Tempo de Processamento]{{
        \begin{tikzpicture}             
            \begin{semilogyaxis}[
                boxplot,
                table/y=TEnd,
                boxplot/draw direction=y,
                %title={Tempo total de execução},
                xlabel={Métodos},
                ylabel={Tempo(s)},
                grid=major,
                xtick={1,2,3,4},
                xticklabels={DFP,BFGS,Huang,Biggs},]
                \addplot table{\tblADFP};
                \addplot table{\tblABFGS};
                \addplot table{\tblAHuang};
                \addplot table{\tblABiggs};
            \end{semilogyaxis}    
        \end{tikzpicture} 
        }}%
    \qquad
    \subfloat[\centering Numero de Iterações]{{
        \begin{tikzpicture}
            \begin{semilogyaxis}[
                boxplot,
                table/y=N_Iter,
                boxplot/draw direction=y,
                %title={},
                xlabel={Métodos},
                ylabel={Iterações(-)},
                grid=major,
                xtick={1,2,3,4},
                xticklabels={DFP,BFGS,Huang,Biggs},]
                \addplot table{\tblADFP};
                \addplot table{\tblABFGS};
                \addplot table{\tblAHuang};
                \addplot table{\tblABiggs};
            \end{semilogyaxis}    
        \end{tikzpicture} 
        }}%
    \qquad
    \subfloat[\centering Número de Avaliações de $f(x)$]{{
        \begin{tikzpicture}
            \begin{semilogyaxis}[
                boxplot,
                table/y=N_Evals_F,
                boxplot/draw direction=y,
                %title={Numero de avaliações da função Objetivo},
                xlabel={Métodos},
                ylabel={Avaliações(-)},
                grid=major,
                xtick={1,2,3,4},
                xticklabels={DFP,BFGS,Huang,Biggs},]
                \addplot table{\tblADFP};
                \addplot table{\tblABFGS};
                \addplot table{\tblAHuang};
                \addplot table{\tblABiggs};
            \end{semilogyaxis}   
        \end{tikzpicture} 
    }}%
    \caption{Análise gráfica da dispersão dos Métodos Quase-Newton para a técnica da seção áurea feita por meio da avaliação direta da função objetivo e parâmetro $\alpha=-0.0263$.}%
    \label{fig:quanewaudiraneg}%
\end{figure}

\FloatBarrier
\newpage

\pgfplotsset{width=18cm,height=4.94cm}
 \begin{figure}[h!]
    \centering            
    \subfloat[\centering Erro $X_{sol}$ (\%)]{{
        \begin{tikzpicture}
            \begin{axis}[%
            grid=major,
            xmax=60,
            scatter/classes={%
            DFP={mark=diamond*,draw=black},BFGS={mark=square*,draw=blue},Huang={mark=star,draw=red},Biggs={mark=triangle*,draw=green}}]
                \addplot[scatter,only marks,scatter src=explicit symbolic]table[x=spl,y=Erro_Xsol_P,meta=Metodo]{\tblADFP};
                \addplot[scatter,only marks,scatter src=explicit symbolic]table[x=spl,y=Erro_Xsol_P,meta=Metodo]{\tblABFGS};
                \addplot[scatter,only marks,scatter src=explicit symbolic]table[x=spl,y=Erro_Xsol_P,meta=Metodo]{\tblAHuang};
                \addplot[scatter,only marks,scatter src=explicit symbolic]table[x=spl,y=Erro_Xsol_P,meta=Metodo]{\tblABiggs};
                \addlegendentry{DFP}
                \addlegendentry{BFGS}
                \addlegendentry{Huang}
                \addlegendentry{Biggs}
            \end{axis}                     
        \end{tikzpicture} 
        }}%   
    \qquad
    \subfloat[\centering Erro $F(x_{sol})$ (\%)]{{
        \begin{tikzpicture}
            \begin{axis}[%
            grid=major,
            xmax=60,
            scatter/classes={%
            DFP={mark=diamond*,draw=black},BFGS={mark=square*,draw=blue},Huang={mark=star,draw=red},Biggs={mark=triangle*,draw=green}}]
                \addplot[scatter,only marks,scatter src=explicit symbolic]table[x=spl,y=Erro_F_P,meta=Metodo]{\tblADFP};
                \addplot[scatter,only marks,scatter src=explicit symbolic]table[x=spl,y=Erro_F_P,meta=Metodo]{\tblABFGS};
                \addplot[scatter,only marks,scatter src=explicit symbolic]table[x=spl,y=Erro_F_P,meta=Metodo]{\tblAHuang};
                \addplot[scatter,only marks,scatter src=explicit symbolic]table[x=spl,y=Erro_F_P,meta=Metodo]{\tblABiggs};
                \addlegendentry{DFP}
                \addlegendentry{BFGS}
                \addlegendentry{Huang}
                \addlegendentry{Biggs}
            \end{axis}                     
        \end{tikzpicture} 
        }}%    
    \caption{{Análise gráfica dos erros dos Métodos Quase-Newton para a técnica da seção áurea feita por meio da avaliação direta da função objetivo e parâmetro $\alpha=-0.0263$.}}%
    \label{fig:example}%
\end{figure} 

    \subsubsection{Conclusão Parcial}
        Para funções quadráticas, os métodos Quase-Newton:
        \begin{itemize}
        \item {Tempo (s):} 
        \item {Iterações:} 
        \item {Avaliações de$F(x)$:} 
        \item {Erro $X_{sol}$ (\%):} 
        \item {Erro $F(x_{sol})$ (\%):} 
        \end{itemize}

\FloatBarrier
\newpage