% Experimentos e Resultados

\pgfplotstableread{data/sect3s2/sec3s2tbl11a.txt}{\tblk}
\pgfplotstableread{data/sect3s2/sec3s2tbl11b.txt}{\tbll}

\pgfplotstableread{data/sect3s2/sect3s2f2_TEnd_Medio.txt}{\tblanovaT}
\pgfplotstableread{data/sect3s2/sect3s2f2_N_Iter_Medio.txt}{\tblanovaNI}
\pgfplotstableread{data/sect3s2/sect3s2f2_N_Evals_F_Medio.txt}{\tblanovaNE}
\pgfplotstableread{data/sect3s2/sect3s2f2_Erro_Xsol_P_Medio.txt}{\tblanovaEX}
\pgfplotstableread{data/sect3s2/sect3s2f2_Erro_F_P_Medio.txt}{\tblanovaEF}

\subsection{Segunda função analisada: $\alpha = +0.0263$}

\begin{minipage}[h!]{\linewidth}
    \centering
    {Aproximações Quadráticas de $f(x)$}
    \label{tab:tblk} 
    \writetable{\tblk}\par
    \bigskip
    \centering
    {Avaliação Direta de $f(x)$}
    \label{tab:tbll} 
    \writetable{\tbll}
    \captionof{table}{Resultados relacionados ao esforço computacional e precisão considerando $\alpha = 0.0263$}
\end{minipage}

\begin{minipage}[h!]{\linewidth}
    \centering
    \hrule
    \vspace{2mm}
    {Tabela Anova de 2 Fatores com blocos Aleatorizados}
    \vspace{2mm}
    \noindent
    \hrule 
    \vspace{2mm}
    Tempo de Processamento Médio\\
    \label{tab:tblDa} 
    \writeanova{\tblanovaT}\par
    \bigskip
    \centering
    Número de Iterações\\
    \label{tab:tblDb} 
    \writeanova{\tblanovaNI}\par
    \bigskip
    \centering
    Número de Avaliações de $f(x)$\\
    \label{tab:tblDc} 
    \writeanova{\tblanovaNE}\par
    \bigskip
    \centering
    {Erro de $X_{sol}(\%)$}\\
    \label{tab:tblDb} 
    \writeanova{\tblanovaEX}\par
    \bigskip
    \centering
    {Erro de $F(X_{sol})(\%)$}\\
    \label{tab:tblDb} 
    \writeanova{\tblanovaEF}\par
    \vspace{2mm}
    \hrule
    \vspace{2mm}
    \captionof{table}{Análise de variância dos Métodos Quase-Newton e da técnica da seção áurea para a primeira função analisada do primeiro experimento}
\end{minipage}

\subsubsection{Conclusão Parcial}
        Para funções quadráticas, os métodos Quase-Newton:
        \begin{itemize}
            \item {Tempo (s):} O teste F da tabela anova indica que o método de cálculo da Seção Áurea interfere significativamente para a variação dos tempos médios de execução. A observação dos resultados aponta que a média dos tempos de execução é menor quando se empregam as aproximações Quadráticas para $F(s)$.
            \item {Iterações:} O teste F da tabela anova indica que ambos, o método de otimização e o método de cálculo da Seção Áurea interferem significativamente para a variação dos número de iterações do algoritmo. Sendo o método de cálculo da seção áurea o fator mais significativo. A observação dos resultados aponta que o número de iterações é menor quando se empregam as aproximações Quadráticas para $F(s)$. No melhor caso, os algoritmos DFP e BFGS performam melhor para qualquer uma das escolhas do método de cálculo da seção áurea. No pior caso, os métodos de Huang e Biggs tem a pior performance quando combinados com a avaliação direta de $F(x)$
            \item {Avaliações de $F(x)$:} O teste F da tabela anova indica que ambos, o método de otimização e o método de cálculo da Seção Áurea interferem significativamente para a variação dos número de Avaliações de $F(x)$. Sendo o método de cálculo da seção áurea o fator mais significativo. Para o melhor caso, o emprego de aproximações quadráticas de $F(x)$ obtém o menor valor para todas os métodos de otimização. Para o pior caso, a combinação do método de Huang com a avaliação direta de $F(x)$ obtém o maior valor. A relação entre o melhor e o pior é de 73 vezes.
            \item {Erro $X_{sol}$ (\%):} O teste F da tabela anova indica que ambos, o método de otimização e o método de cálculo da Seção Áurea interferem significativamente para a variação do erro $X_{sol}$. Sendo o método de cálculo da seção áurea o fator mais significativo. Para o melhor caso, o emprego de aproximações quadráticas de $F(x)$ combinado com o método de Huang obtém o menor valor. Para o pior caso, a combinação da avaliação direta de $F(x)$ com qualquer método de otimização obtém o maior valor.
            \item {Erro $F(x_{sol})$ (\%):} \textbf{Sempre} zeram o erro da Função objetivo calculado no ponto da solução para todas variações de todos os fatores, o valor obtido $10^{-16}$ é praticamente zero pois é da mesma ordem de valor da precisão relativa do ponto flutuante no Matlab. O teste F da tabela anova considera que nenhum dos fatores é capaz de causar alteração significativa do Erro $F(x_{sol})$ (\%), devido a esta variável ter valor constante e igual a 0 para todos os blocos executados.
        \end{itemize}