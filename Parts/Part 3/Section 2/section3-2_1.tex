% Experimentos e Resultados

\pgfplotstableread{data/sect3s2/sec3s2tbl10a.txt}{\tbli}
\pgfplotstableread{data/sect3s2/sec3s2tbl10b.txt}{\tblj}

\pgfplotstableread{data/sect3s2/sect3s2f1_TEnd_Medio.txt}{\tblanovaT}
\pgfplotstableread{data/sect3s2/sect3s2f1_N_Iter_Medio.txt}{\tblanovaNI}
\pgfplotstableread{data/sect3s2/sect3s2f1_N_Evals_F_Medio.txt}{\tblanovaNE}
\pgfplotstableread{data/sect3s2/sect3s2f1_Erro_Xsol_P_Medio.txt}{\tblanovaEX}
\pgfplotstableread{data/sect3s2/sect3s2f1_Erro_F_P_Medio.txt}{\tblanovaEF}

\subsection{Primeira função analisada: $\alpha = -0.0263$}

\begin{minipage}[h!]{\linewidth}
    \centering
    Aproximações Quadráticas de $f(x)$
    \label{tab:tbli} 
    \writetable{\tbli}\par
    \bigskip
    \centering
    Avaliação Direta de $f(x)$
    \label{tab:tblj} 
    \writetable{\tblj}
    \captionof{table}{Resultados relacionados ao esforço computacional e precisão considerando $\alpha = -0.0263$}
\end{minipage}

\begin{minipage}[h!]{\linewidth}
    \centering
    \hrule
    \vspace{2mm}
    {Tabela Anova de 2 Fatores com blocos Aleatorizados}
    \vspace{2mm}
    \noindent
    \hrule 
    \vspace{2mm}
    Tempo de Processamento Médio\\
    \label{tab:tblDa} 
    \writeanova{\tblanovaT}\par
    \bigskip
    \centering
    Número de Iterações\\
    \label{tab:tblDb} 
    \writeanova{\tblanovaNI}\par
    \bigskip
    \centering
    Número de Avaliações de $f(x)$\\
    \label{tab:tblDc} 
    \writeanova{\tblanovaNE}\par
    \bigskip
    \centering
    {Erro de $X_{sol}(\%)$}\\
    \label{tab:tblDb} 
    \writeanova{\tblanovaEX}\par
    \bigskip
    \centering
    {Erro de $F(X_{sol})(\%)$}\\
    \label{tab:tblDb} 
    \writeanova{\tblanovaEF}\par
    \vspace{2mm}
    \hrule
    \vspace{2mm}
    \captionof{table}{Análise de variância dos Métodos Quase-Newton e da técnica da seção áurea para a primeira função analisada do primeiro experimento}
\end{minipage}

\subsubsection{Conclusão Parcial}
        Para funções quadráticas, os métodos Quase-Newton:
        \begin{itemize}
            \item {Tempo (s):} O teste F da tabela anova indica que ambos, o método ode otimização e o método de cálculo da Seção Áurea interferem significativamente para a variação dos tempos médios de execução. Sendo o método de cálculo da seção áurea o fator mais significativo. A observação dos resultados aponta que o melhor caso se obtém empregando as aproximações Quadráticas para $F(s)$ apresentam um tempo médio de execução 2 vezes menor. Entretanto a observação dos dados mostra também que apesar do desempenho médio empregando aproximações Quadráticas para $F(s)$ no cálculo da seção áurea, o menor tempo de execução se deu pela combinação do Método de otimização de BFGS e a avaliação direta de $F(s)$. Enquanto isso, o pior caso ocorreu quando combina-se o método de otimização de Huang e a avaliação direta de $F(s)$. 
            \item {Iterações:}  O teste F da tabela anova indica que ambos, o método de otimização e o método de cálculo da Seção Áurea interferem significativamente para a variação dos número de iterações do algoritmo. Sendo o método de cálculo da seção áurea o fator mais significativo. A observação dos resultados aponta que o melhor caso se dá pela combinação dos métodos DFP ou BFGS e a avaliação direta de $F(x)$. Da mesma maneira, o pior caso se dá pela combinação dos dos métodos Huang ou Biggs e Aproximações quadráticas de $F(x)$.
            \item {Avaliações de $F(x)$:} O teste F da tabela anova indica que ambos, o método de otimização e o método de cálculo da Seção Áurea interferem significativamente para a variação dos número de avaliações de $F(x)$. Sendo o método de cálculo da seção áurea o fator mais significativo. A observação dos resultados aponta que o melhor caso se dá pela combinação dos métodos DFP ou BFGS e aproximações quadráticas de $F(x)$. Da mesma maneira, o pior caso se dá pela combinação dos dos métodos Huang ou Biggs e a avaliação direta de $F(x)$.
            \item {Erro $X_{sol}$ (\%):} O teste F da tabela anova indica que ambos, o método de otimização e o método de cálculo da Seção Áurea interferem significativamente para a variação do erro $X_{sol}$ (\%). Sendo o método de cálculo da seção áurea o fator mais significativo. A observação dos resultados aponta que o melhor caso se dá pela combinação dos dos métodos DFP e aproximações quadráticas de $F(x)$. Da mesma maneira, o pior caso se dá pela combinação dos dos métodos DFP ou BFGS e a avaliação direta de $F(x)$. Pela observação dos dados os métodos interferem mais na variação do erro $X_{sol}$ (\%) de forma independente ao método de cálculo da seção áurea.
            \item {Erro $F(x_{sol})$ (\%):} O teste F da tabela anova indica que ambos, o método de otimização e o método de cálculo da Seção Áurea interferem significativamente para a variação do erro $F(x_{sol})$ (\%). Sendo o método de cálculo da seção áurea o fator mais significativo. A observação dos resultados aponta que o melhor caso se dá pela combinação dos dos métodos Huang ou Biggs e aproximações quadráticas de $F(x)$ que são capazes de zerar por completo o erro. Todas as outras combinações tem o mesmo efeito sobre o erro $F(x_{sol})$ (\%), tornando-o da ordem de $10^{-15}$.
        \end{itemize}