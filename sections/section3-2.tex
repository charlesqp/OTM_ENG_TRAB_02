% Experimentos e Resultados

\pgfplotstableread{data/sect3s2/sec3s2tbl10a.txt}{\tbli}
\pgfplotstableread{data/sect3s2/sec3s2tbl10b.txt}{\tblj}
\pgfplotstableread{data/sect3s2/sec3s2tbl11a.txt}{\tblk}
\pgfplotstableread{data/sect3s2/sec3s2tbl11b.txt}{\tbll}
\pgfplotstableread{data/sect3s2/sec3s2tbl12a.txt}{\tblm}
\pgfplotstableread{data/sect3s2/sec3s2tbl12b.txt}{\tbln}

\section{2° Experimento: Avaliação das Funções Não Quadráticas}
    \subsection{Função Analisada}

    A função analisada consiste na função da seção \ref{sec:secfun}.

    A função objetivo apresentada possui duas variáveis de decisão. Nesse caso, o número máximo de iterações, calculado por meio da equação \ref{eq:maxevals}, é igual a 400 e os limites das variáveis de decisão definem o seguinte intervalo $[\ -10 , 10 ]\ $. No intuito de obter uma função não quadrática perfeita, foram considerados três valores do parâmetro $a$, os quais são apresentados abaixo seguido da respectiva função objetivo:

    \begin{equation*}   
        \begin{aligned}
            \alpha &=-0.0263 \rightarrow &f(x) = 12*x_1^2 - 4*x_2^2 - 12*x_1x_2 + 2*x_1 -0.0263*(x_1^3+x_2^3)\\
            \alpha &=+0.0263 \rightarrow &f(x) = 12*x_1^2 - 4*x_2^2 - 12*x_1x_2 + 2*x_1 + 0.0263*(x_1^3+x_2^3)\\
            \alpha &=1 \rightarrow &f(x) = 12*x_1^2 - 4*x_2^2 - 12*x_1x_2 + 2*x_1 + (x_1^3+x_2^3)\\   
        \end{aligned}
    \end{equation*}
 
    Para cada valor do parâmetro $a$ deverão ser considerados dois pontos iniciais, dentre os quais um será executado considerando a técnica da seção áurea feita através da avaliação direta da função e o outro considerando as aproximações quadráticas para a função a cada iteração.

    Vale ressaltar que o parâmetro a pondera o termo cúbico da função objetivo $f(x)$ do problema de otimização estudado, quanto maior o valor absoluto desse parâmetro, maior é a influência do termo não quadrático na função. Para esta função os valores assumidos para parâmetro  a que não comprometem a convexidade da função   no intervalo $[\ -10 , 10 ]\ $ estão no intervalo: \\ {\centering $-0.0263 \leq a \leq 0.0263$} \\
    \vspace{2mm}

    {\centering
    \hrule
    \vspace{2mm}
    $\alpha = -0.0263$
    \vspace{2mm}
    \noindent
    \hrule 
    \vspace{2mm}}
      
    \begin{minipage}[h!]{\linewidth}
        \centering
        Aproximações Quadráticas de $f(x)$
        \label{tab:tbli} 
        \writetable{\tbli}\par
        \bigskip
        \centering
        Avaliação Direta de $f(x)$
        \label{tab:tblj} 
        \writetable{\tblj}
        \captionof{table}{Resultados relacionados ao esforço computacional e precisão considerando $\alpha = -0.0263$}
    \end{minipage}
    \vspace{3mm}

    {\centering
    \hrule
    \vspace{2mm}
    $\alpha = 0.0263$
    \vspace{2mm}
    \noindent
    \hrule 
    \vspace{2mm}}
    
    \begin{minipage}[h!]{\linewidth}
        \centering
        {Aproximações Quadráticas de $f(x)$}
        \label{tab:tblk} 
        \writetable{\tblk}\par
        \bigskip
        \centering
        {Avaliação Direta de $f(x)$}
        \label{tab:tbll} 
        \writetable{\tbll}
        \captionof{table}{Resultados relacionados ao esforço computacional e precisão considerando $\alpha = 0.0263$}
    \end{minipage}
    \vspace{3mm}

    {\centering
    \hrule
    \vspace{2mm}
    $\alpha = 1.0$
    \vspace{2mm}
    \noindent
    \hrule 
    \vspace{2mm}}
    
    \begin{minipage}[h!]{\linewidth}
        \centering
        {Aproximações Quadráticas de $f(x)$}
        \label{tab:tblm} 
        \writetable{\tblm}\par
        \bigskip
        \centering
        {Avaliação Direta de $f(x)$}
        \label{tab:tbln} 
        \writetable{\tbln}     
        \captionof{table}{Resultados relacionados ao esforço computacional e precisão considerando $\alpha = 1.0$}
    \end{minipage}
    \vspace{3mm}

    \subsubsection{Conclusão Parcial}
            Para funções quadráticas, os métodos Quase-Newton:
            \begin{itemize}
            \item {Tempo (s):} 
            \item {Iterações:} 
            \item {Avaliações de$F(x)$:} 
            \item {Erro $X_{sol}$ (\%):} 
            \item {Erro $F(x_{sol})$ (\%):} 
            \end{itemize}
         
\newpage    