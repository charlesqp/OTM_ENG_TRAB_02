% Experimentos e Resultados

\pgfplotstableread{data/sect3s3/sec3s3tbl13a.txt}{\tblo}
\pgfplotstableread{data/sect3s3/sec3s3tbl13b.txt}{\tblp}
\pgfplotstableread{data/sect3s3/sec3s3tbl13c.txt}{\tblq}
\pgfplotstableread{data/sect3s3/sec3s3tbl13d.txt}{\tblr}
\pgfplotstableread{data/sect3s3/sec3s3tbl13e.txt}{\tbls}

\section{3° Experimento: Avaliação do Custo Computacional em Função de Variação da Dimensão}
    \subsection{Função Analisada}

        \begin{minipage}[h!]{\linewidth}
            \centering
            Dimensão igual a 5
            \label{tab:tblo} 
            \writetable{\tblo}\par
            \bigskip
            \centering
            Dimensão igual a 10
            \label{tab:tblp} 
            \writetable{\tblp}\par
            \bigskip
            \centering
            Dimensão igual a 15
            \label{tab:tblq} 
            \writetable{\tblq}\par
            \bigskip
            \centering
            Dimensão igual a 20
            \label{tab:tblr} 
            \writetable{\tblr}\par
            \bigskip
            \centering
            Dimensão igual a 25
            \label{tab:tbls} 
            \writetable{\tbls}
            \captionof{table}{Resultados relacionados ao esforço computacional e precisão considerando a dimensão $n$}
        \end{minipage}

        \subsubsection{Conclusão Parcial}
            Para funções quadráticas, os métodos Quase-Newton:
            \begin{itemize}
            \item {Tempo (s):} 
            \item {Iterações:} 
            \item {Avaliações de$F(x)$:} 
            \item {Erro $X_{sol}$ (\%):} 
            \item {Erro $F(x_{sol})$ (\%):} 
            \end{itemize}
    
\newpage