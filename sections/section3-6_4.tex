\pgfplotstableread{data/sect3s6s4/sec1s1tbl20_tEnd.txt}{\tblDa}
\pgfplotstableread{data/sect3s6s4/sec1s1tbl20_N_Iter.txt}{\tblDb}
\pgfplotstableread{data/sect3s6s4/sec1s1tbl20_N_Evals.txt}{\tblDc}
\pgfplotstableread{data/sect3s6s4/sec1s1tbl20_Erro_Xsol_P.txt}{\tblDd}
\pgfplotstableread{data/sect3s6s4/sec1s1tbl20_Erro_F_P.txt}{\tblDe}
\pgfplotstableread{data/sect3s6s4/sec1s1tbl20_DFP_raw.txt}{\tblDDFP}
\pgfplotstableread{data/sect3s6s4/sec1s1tbl20_BFGS_raw.txt}{\tblDBFGS}
\pgfplotstableread{data/sect3s6s4/sec1s1tbl20_Huang_raw.txt}{\tblDHuang}
\pgfplotstableread{data/sect3s6s4/sec1s1tbl20_Biggs_raw.txt}{\tblDBiggs}

\begin{minipage}[h!]{\linewidth}
    \centering
    \hrule
    \vspace{2mm}
    Técnica da Seção Áurea Feita por meio de Aproximações Quadráticas para $F(x)$ \\ $\alpha=0.0236$
    \vspace{2mm}
    \noindent
    \hrule 
    \vspace{2mm}
    Tempo de Processamento Médio\\
    \label{tab:tblDa} 
    \writetablestt{\tblDa}\par
    \bigskip
    \centering
    Número de Iterações\\
    \label{tab:tblDb} 
    \writetablestt{\tblDb}\par
    \bigskip
    \centering
    Número de Avaliações de $f(x)$\\
    \label{tab:tblDc} 
    \writetablestt{\tblDc}\par
    \vspace{2mm}
    \hrule
    \vspace{2mm}
    \captionof{table}{Análise estatística dos Métodos Quase-Newton para a técnica da seção áurea feita por meio da avaliação direta da função objetivo e parâmetro $\alpha=0.0263$}
\end{minipage}


\pgfplotsset{width=7.5cm,height=6cm,compat=1.18}\usepgfplotslibrary{statistics} 
\begin{figure}[h!]
    \centering            
    \subfloat[\centering Tempo de Processamento]{{
        \begin{tikzpicture}
            \begin{semilogyaxis}[
                boxplot,
                table/y=TEnd,
                boxplot/draw direction=y,
                %title={Tempo total de execução},
                xlabel={Métodos},
                ylabel={Tempo(s)},
                grid=major,
                xtick={1,2,3,4},
                xticklabels={DFP,BFGS,Huang,Biggs},]
                \addplot table{\tblDDFP};
                \addplot table{\tblDBFGS};
                \addplot table{\tblDHuang};
                \addplot table{\tblDBiggs};
            \end{semilogyaxis}    
        \end{tikzpicture} 
        }}%
    \qquad
    \subfloat[\centering Numero de Iterações]{{
        \begin{tikzpicture}
            \begin{semilogyaxis}[
                boxplot,
                table/y=N_Iter,
                boxplot/draw direction=y,
                %title={},
                xlabel={Métodos},
                ylabel={Iterações(-)},
                grid=major,
                xtick={1,2,3,4},
                xticklabels={DFP,BFGS,Huang,Biggs},]
                \addplot table{\tblDDFP};
                \addplot table{\tblDBFGS};
                \addplot table{\tblDHuang};
                \addplot table{\tblDBiggs};
            \end{semilogyaxis}    
        \end{tikzpicture} 
        }}%
    \qquad
    \subfloat[\centering Número de Avaliações de $f(x)$]{{
        \begin{tikzpicture}
            \begin{semilogyaxis}[
                boxplot,
                table/y=N_Evals_F,
                boxplot/draw direction=y,
                %title={Numero de avaliações da função Objetivo},
                xlabel={Métodos},
                ylabel={Avaliações(-)},
                grid=major,
                xtick={1,2,3,4},
                xticklabels={DFP,BFGS,Huang,Biggs},]
                \addplot table{\tblDDFP};
                \addplot table{\tblDBFGS};
                \addplot table{\tblDHuang};
                \addplot table{\tblDBiggs};
            \end{semilogyaxis}   
            \label{fig:fig1c}%
        \end{tikzpicture} 
    }}%
    \caption{Análise gráfica da dispersão dos Métodos Quase-Newton para a técnica da seção áurea feita por meio da avaliação direta da função objetivo e parâmetro $\alpha=0.0263$.}%
    \label{fig:fig1}%
\end{figure}

\FloatBarrier
\newpage

\pgfplotsset{width=18cm,height=4.94cm}
 \begin{figure}[h!]
    \centering            
    \subfloat[\centering Erro $X_{sol}$ (\%)]{{
        \begin{tikzpicture}
            \begin{axis}[%
            grid=major,
            xmax=60,
            scatter/classes={%
            DFP={mark=diamond*,draw=black},BFGS={mark=square*,draw=blue},Huang={mark=star,draw=red},Biggs={mark=triangle*,draw=green}}]
                \addplot[scatter,only marks,scatter src=explicit symbolic]table[x=spl,y=Erro_Xsol_P,meta=Metodo]{\tblDDFP};
                \addplot[scatter,only marks,scatter src=explicit symbolic]table[x=spl,y=Erro_Xsol_P,meta=Metodo]{\tblDBFGS};
                \addplot[scatter,only marks,scatter src=explicit symbolic]table[x=spl,y=Erro_Xsol_P,meta=Metodo]{\tblDHuang};
                \addplot[scatter,only marks,scatter src=explicit symbolic]table[x=spl,y=Erro_Xsol_P,meta=Metodo]{\tblDBiggs};
                \addlegendentry{DFP}
                \addlegendentry{BFGS}
                \addlegendentry{Huang}
                \addlegendentry{Biggs}
            \end{axis}                     
        \end{tikzpicture} 
        }}%   
    \qquad
    \subfloat[\centering Erro $F(x_{sol})$ (\%)]{{
        \begin{tikzpicture}
            \begin{axis}[%
            grid=major,
            xmax=60,
            scatter/classes={%
            DFP={mark=diamond*,draw=black},BFGS={mark=square*,draw=blue},Huang={mark=star,draw=red},Biggs={mark=triangle*,draw=green}}]
                \addplot[scatter,only marks,scatter src=explicit symbolic]table[x=spl,y=Erro_F_P,meta=Metodo]{\tblDDFP};
                \addplot[scatter,only marks,scatter src=explicit symbolic]table[x=spl,y=Erro_F_P,meta=Metodo]{\tblDBFGS};
                \addplot[scatter,only marks,scatter src=explicit symbolic]table[x=spl,y=Erro_F_P,meta=Metodo]{\tblDHuang};
                \addplot[scatter,only marks,scatter src=explicit symbolic]table[x=spl,y=Erro_F_P,meta=Metodo]{\tblDBiggs};
                \addlegendentry{DFP}
                \addlegendentry{BFGS}
                \addlegendentry{Huang}
                \addlegendentry{Biggs}
            \end{axis}                     
        \end{tikzpicture} 
        }}%    
    \caption{{Análise gráfica dos erros dos Métodos Quase-Newton para a técnica da seção áurea feita por meio da avaliação direta da função objetivo e parâmetro $\alpha=0.0263$.}}%
    \label{fig:example}%
\end{figure} 

\FloatBarrier
\newpage
