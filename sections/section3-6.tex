% Experimentos e Resultados

\pgfplotstableread{data/sec1s1tbl17_tEnd.txt}{\tblAa}
\pgfplotstableread{data/sec1s1tbl17_N_Iter.txt}{\tblAb}
\pgfplotstableread{data/sec1s1tbl17_N_Evals.txt}{\tblAc}
\pgfplotstableread{data/sec1s1tbl17_Erro_Xsol_P.txt}{\tblAd}
\pgfplotstableread{data/sec1s1tbl17_Erro_F_P.txt}{\tblAe}
\pgfplotstableread{data/sec1s1tbl17_DFP_raw.txt}{\tblADFP}
\pgfplotstableread{data/sec1s1tbl17_BFGS_raw.txt}{\tblABFGS}
\pgfplotstableread{data/sec1s1tbl17_Huang_raw.txt}{\tblAHuang}
\pgfplotstableread{data/sec1s1tbl17_Biggs_raw.txt}{\tblABiggs}

\pgfplotstableread{data/sec1s1tbl18_tEnd.txt}{\tblBa}
\pgfplotstableread{data/sec1s1tbl18_N_Iter.txt}{\tblBb}
\pgfplotstableread{data/sec1s1tbl18_N_Evals.txt}{\tblBc}
\pgfplotstableread{data/sec1s1tbl18_Erro_Xsol_P.txt}{\tblBd}
\pgfplotstableread{data/sec1s1tbl18_Erro_F_P.txt}{\tblBe}
\pgfplotstableread{data/sec1s1tbl18_DFP_raw.txt}{\tblBDFP}
\pgfplotstableread{data/sec1s1tbl18_BFGS_raw.txt}{\tblBBFGS}
\pgfplotstableread{data/sec1s1tbl18_Huang_raw.txt}{\tblBHuang}
\pgfplotstableread{data/sec1s1tbl18_Biggs_raw.txt}{\tblBBiggs}

\pgfplotstableread{data/sec1s1tbl19_tEnd.txt}{\tblCa}
\pgfplotstableread{data/sec1s1tbl19_N_Iter.txt}{\tblCb}
\pgfplotstableread{data/sec1s1tbl19_N_Evals.txt}{\tblCc}
\pgfplotstableread{data/sec1s1tbl19_Erro_Xsol_P.txt}{\tblCd}
\pgfplotstableread{data/sec1s1tbl19_Erro_F_P.txt}{\tblCe}

\pgfplotstableread{data/sec1s1tbl20_tEnd.txt}{\tblDa}
\pgfplotstableread{data/sec1s1tbl20_N_Iter.txt}{\tblDb}
\pgfplotstableread{data/sec1s1tbl20_N_Evals.txt}{\tblDc}
\pgfplotstableread{data/sec1s1tbl20_Erro_Xsol_P.txt}{\tblDd}
\pgfplotstableread{data/sec1s1tbl20_Erro_F_P.txt}{\tblDe}

\subsection{6° Experimento: Avaliação Estatística dos Métodos Quase-Newton com Variação do Parâmetro $\alpha$}
    \subsubsection{Função Analisada}

        \begin{minipage}[h!]{\linewidth}            
            \centering
            \hrule
            \vspace{2mm}
            Técnica da Seção Áurea Feita por meio da avaliação direta de $F(x)$ \\ $\alpha=-0.0236$
            \vspace{2mm}
            \noindent
            \hrule 
            \vspace{2mm}
            Tempo de Processamento Médio\\
            \label{tab:tblAa} 
            \writetablestt{\tblAa}\par
            \bigskip
            \centering
            Número de Iterações\\
            \label{tab:tblAb} 
            \writetablestt{\tblAb}\par
            \bigskip
            \centering
            Número de Avaliações de $f(x)$\\
            \label{tab:tblAc} 
            \writetablestt{\tblAc}\par
            \vspace{2mm}
            \hrule
            \vspace{2mm}
            \captionof{table}{Análise estatística dos Métodos Quase-Newton para a técnica da seção áurea feita por meio da avaliação direta da função objetivo e parâmetro $\alpha=-0.0263$}
        \end{minipage}

             

 
            
        \begin{figure}[h!]            \pgfplotsset{width=8cm,height=4.94cm,compat=1.18}\usepgfplotslibrary{statistics} 
            \centering            
            \subfloat[\centering label 1]{{
                \begin{tikzpicture}
                    \begin{semilogyaxis}[
                        boxplot,
                        table/y=TEnd,
                        boxplot/draw direction=y,
                        title={Tempo total de execução},
                        xlabel={Métodos},
                        grid=major,
                        xtick={1,2,3,4},
                        xticklabels={DFP,BFGS,Huang,Biggs},]
                        \addplot table{\tblADFP};
                        \addplot table{\tblABFGS};
                        \addplot table{\tblAHuang};
                        \addplot table{\tblABiggs};
                    \end{semilogyaxis}    
                \end{tikzpicture} 
                }}%
            \qquad
            \subfloat[\centering label 2]{{
                \begin{tikzpicture}
                    \begin{semilogyaxis}[
                        boxplot,
                        table/y=N_Iter,
                        boxplot/draw direction=y,
                        title={Numero de Iterações},
                        xlabel={Métodos},
                        grid=major,
                        xtick={1,2,3,4},
                        xticklabels={DFP,BFGS,Huang,Biggs},]
                        \addplot table{\tblADFP};
                        \addplot table{\tblABFGS};
                        \addplot table{\tblAHuang};
                        \addplot table{\tblABiggs};
                    \end{semilogyaxis}    
                \end{tikzpicture} 
                }}%
            \qquad
            \subfloat[\centering label 3]{{
                \begin{tikzpicture}
                    \begin{semilogyaxis}[
                        boxplot,
                        table/y=N_Evals_F,
                        boxplot/draw direction=y,
                        title={Numero de avaliações da função Objetivo},
                        xlabel={Métodos},
                        grid=major,
                        xtick={1,2,3,4},
                        xticklabels={DFP,BFGS,Huang,Biggs},]
                        \addplot table{\tblADFP};
                        \addplot table{\tblABFGS};
                        \addplot table{\tblAHuang};
                        \addplot table{\tblABiggs};
                    \end{semilogyaxis}    
                \end{tikzpicture} 
            }}%
            \caption{2 Figures side by side}%
            \label{fig:example}%
        \end{figure}

         \begin{figure}[h!]            
            \pgfplotsset{width=16cm,height=4.94cm}
            \centering            
            \subfloat[\centering label 1]{{
                \begin{tikzpicture}
                    \begin{axis}
                        \addplot[scatter,only marks,scatter src=a,draw=black]table[x=spl,y=Erro_Xsol_P,meta=Metodo]{\tblADFP};
                    \end{axis}                     
                \end{tikzpicture} 
                }}%              
            \caption{2 Figures side by side}%
            \label{fig:example}%
        \end{figure}
            
            

        \begin{minipage}[h!]{\linewidth}
            \centering
            \hrule
            \vspace{2mm}
            Técnica da Seção Áurea Feita por meio de Aproximações Quadráticas para $F(x)$ \\ $\alpha=0$
            \vspace{2mm}
            \noindent
            \hrule 
            \vspace{2mm}
            Tempo de Processamento Médio\\
            \label{tab:tblBa} 
            \writetablestt{\tblBa}\par
            \bigskip
            \centering
            Número de Iterações\\
            \label{tab:tblBb} 
            \writetablestt{\tblBb}\par
            \bigskip
            \centering
            Número de Avaliações de $f(x)$\\
            \label{tab:tblBc} 
            \writetablestt{\tblBc}\par
            \vspace{2mm}
            \hrule
            \vspace{2mm}
            \captionof{table}{Análise estatística dos Métodos Quase-Newton para a técnica da seção áurea feita por meio da avaliação direta da função objetivo e parâmetro $\alpha=0$}
        \end{minipage}

        \begin{minipage}[h!]{\linewidth}
            \centering
            \hrule
            \vspace{2mm}
            Técnica da Seção Áurea Feita por meio da avaliação direta de $F(x)$ \\ $\alpha=0$
            \vspace{2mm}
            \noindent
            \hrule 
            \vspace{2mm}
            Tempo de Processamento Médio\\
            \label{tab:tblCa} 
            \writetablestt{\tblCa}\par
            \bigskip
            \centering
            Número de Iterações\\
            \label{tab:tblCb} 
            \writetablestt{\tblCb}\par
            \bigskip
            \centering
            Número de Avaliações de $f(x)$\\
            \label{tab:tblCc} 
            \writetablestt{\tblCc}\par
            \vspace{2mm}
            \hrule
            \vspace{2mm}
            \captionof{table}{Análise estatística dos Métodos Quase-Newton para a técnica da seção áurea feita por meio da avaliação direta da função objetivo e parâmetro $\alpha=0$}
        \end{minipage}

        \begin{minipage}[h!]{\linewidth}
            \centering
            \hrule
            \vspace{2mm}
            Técnica da Seção Áurea Feita por meio de Aproximações Quadráticas para $F(x)$ \\ $\alpha=0.0236$
            \vspace{2mm}
            \noindent
            \hrule 
            \vspace{2mm}
            Tempo de Processamento Médio\\
            \label{tab:tblDa} 
            \writetablestt{\tblDa}\par
            \bigskip
            \centering
            Número de Iterações\\
            \label{tab:tblDb} 
            \writetablestt{\tblDb}\par
            \bigskip
            \centering
            Número de Avaliações de $f(x)$\\
            \label{tab:tblDc} 
            \writetablestt{\tblDc}\par
            \vspace{2mm}
            \hrule
            \vspace{2mm}
            \captionof{table}{Análise estatística dos Métodos Quase-Newton para a técnica da seção áurea feita por meio da avaliação direta da função objetivo e parâmetro $\alpha=0.0263$}
        \end{minipage}

\newpage