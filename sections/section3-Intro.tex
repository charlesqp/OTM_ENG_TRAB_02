Nesse momento executa-se os algoritmos utilizados para obtenção dos resultados que solucionam os problemas de programação não linear compostos por funções objetivos quadráticas e não quadráticas perfeitas. Diversos estudos da aplicação dos métodos Quase-Newton para a solução desses problemas foram realizados, entre os quais se destacam as análises relacionadas ao esforço computacional dos algoritmos e precisão. O primeiro envolve o número de iterações, o número de avaliações da função objetivo e o tempo de processamento. Já o segundo envolve o erro percentual da solução encontrada e do valor de função objetivo dessa solução. A seguir são apresentados os experimentos e resultados.