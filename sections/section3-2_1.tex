% Experimentos e Resultados

\pgfplotstableread{data/sect3s2/sec3s2tbl10a.txt}{\tbli}
\pgfplotstableread{data/sect3s2/sec3s2tbl10b.txt}{\tblj}

\pgfplotstableread{data/sect3s2/sect3s2f1_TEnd_Medio.txt}{\tblanovaT}
\pgfplotstableread{data/sect3s2/sect3s2f1_N_Iter_Medio.txt}{\tblanovaNI}
\pgfplotstableread{data/sect3s2/sect3s2f1_N_Evals_F_Medio.txt}{\tblanovaNE}
\pgfplotstableread{data/sect3s2/sect3s2f1_Erro_Xsol_P_Medio.txt}{\tblanovaEX}
\pgfplotstableread{data/sect3s2/sect3s2f1_Erro_F_P_Medio.txt}{\tblanovaEF}

\begin{minipage}[h!]{\linewidth}
    \centering
    Aproximações Quadráticas de $f(x)$
    \label{tab:tbli} 
    \writetable{\tbli}\par
    \bigskip
    \centering
    Avaliação Direta de $f(x)$
    \label{tab:tblj} 
    \writetable{\tblj}
    \captionof{table}{Resultados relacionados ao esforço computacional e precisão considerando $\alpha = -0.0263$}
\end{minipage}

\begin{minipage}[h!]{\linewidth}
    \centering
    \hrule
    \vspace{2mm}
    {Tabela Anova de 2 Fatores com blocos Aleatorizados}
    \vspace{2mm}
    \noindent
    \hrule 
    \vspace{2mm}
    Tempo de Processamento Médio\\
    \label{tab:tblDa} 
    \writeanova{\tblanovaT}\par
    \bigskip
    \centering
    Número de Iterações\\
    \label{tab:tblDb} 
    \writeanova{\tblanovaNI}\par
    \bigskip
    \centering
    Número de Avaliações de $f(x)$\\
    \label{tab:tblDc} 
    \writeanova{\tblanovaNE}\par
    \bigskip
    \centering
    {Erro de $X_{sol}(\%)$}\\
    \label{tab:tblDb} 
    \writeanova{\tblanovaEX}\par
    \bigskip
    \centering
    {Erro de $F(X_{sol})(\%)$}\\
    \label{tab:tblDb} 
    \writeanova{\tblanovaEF}\par
    \vspace{2mm}
    \hrule
    \vspace{2mm}
    \captionof{table}{Análise de variância dos Métodos Quase-Newton e da técnica da seção áurea para a primeira função analisada do primeiro experimento}
\end{minipage}

\subsubsection{Conclusão Parcial}
        Para funções quadráticas, os métodos Quase-Newton:
        \begin{itemize}
        \item {Tempo (s):} 
        \item {Iterações:} 
        \item {Avaliações de $F(x)$:} 
        \item {Erro $X_{sol}$ (\%):} 
        \item {Erro $F(x_{sol})$ (\%):} 
        \end{itemize}