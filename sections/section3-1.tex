% Experimentos e Resultados

\pgfplotstableread{data/sect3s1/sec3s1tbl2.txt}{\tbla}
\pgfplotstableread{data/sect3s1/sec3s1tbl3.txt}{\tblb}
\pgfplotstableread{data/sect3s1/sec3s1tbl4.txt}{\tblc}
\pgfplotstableread{data/sect3s1/sec3s1tbl5.txt}{\tbld}
\pgfplotstableread{data/sect3s1/sec3s1tbl6.txt}{\tble}
\pgfplotstableread{data/sect3s1/sec3s1tbl7.txt}{\tblf}
\pgfplotstableread{data/sect3s1/sec3s1tbl8.txt}{\tblg}
\pgfplotstableread{data/sect3s1/sec3s1tbl9.txt}{\tblh}

\section{1° Experimento: Avaliação das Funções Quadráticas}
    \subsection{Primeira Função Analisada}

    A primeira função analisada consiste na função \ref{eq:prifuntst} da seção \ref{sec:prifun}, reapresentada a seguir:

    \begin{equation*} 
        f(x)= \frac{1}{2}*(x-c)^{T}*A*(x-c)
    \end{equation*}

    Para o experimento em questão a função objetivo é avaliada em duas dimensões. Nesse caso, o número máximo de iterações, calculado por meio da equação \ref{eq:maxevals}, é igual a $400$ e os limites inferior e superior das variáveis $[\-10, 10 ]\ $.

    Na execução dos algoritmos, os métodos Quase-Newton, deverão considerados quatro pontos iniciais. Em dois pontos iniciais empregando a técnica da seção áurea diretamente na avaliação da função e, nos outros dois pontos, usando  a técnica da seção áurea feita por aproximações quadráticas para a função, conforme exibido abaixo:
    \begin{itemize}
        \item $x_{0_1}=[\ 9 , 9 ]\ \rightarrow$ 1º Quadrante $\Rightarrow$ Aproximações Quadráticas  
        \item $x_{0_2}=[\ -3 , 2 ]\ \rightarrow$ 2º Quadrante $\Rightarrow$ Avaliação Direta da Função
        \item $x_{0_3}=[\ -8 , -6 ]\ \rightarrow$ 3º Quadrante $\Rightarrow$ Aproximações Quadráticas  
        \item $x_{0_4}=[\ 5 , -7 ]\ \rightarrow$ 4º Quadrante $\Rightarrow$ Avaliação Direta da Função
    \end{itemize} 

    Os resultados obtidos são apresentados nas tabelas abaixo.
    \vspace{2mm}
    
        \begin{minipage}{\linewidth}
            \centering
            $x_{0_{1}}=[9,9]\Longrightarrow$  Aproximações Quadráticas para $f(x)$
            \label{tab:tbla} 
            \writetable{\tbla}
            \bigskip
            \captionof{table}{Resultados relacionados ao esforço computacional e precisão considerando o ponto inicial $x_{0_{1}}$}
        \end{minipage}

        \begin{minipage}{\linewidth}
            \centering
            $x_{0_{2}}=[-3,2]\Longrightarrow$  Avaliação Direta de $f(x)$
            \label{tab:tblb} 
            \writetable{\tblb}
            \bigskip
            \captionof{table}{Resultados relacionados ao esforço computacional e precisão considerando o ponto inicial $x_{0_{2}}$}
        \end{minipage}

        \begin{minipage}{\linewidth}
            \centering
            $x_{0_{3}}=[-8,-6]\Longrightarrow$  Aproximações Quadráticas para $f(x)$
            \label{tab:tblc} 
            \writetable{\tblc}
            \bigskip
            \captionof{table}{Resultados relacionados ao esforço computacional e precisão considerando o ponto inicial $x_{0_{3}}$}
        \end{minipage}
        
        \begin{minipage}{\linewidth}
            \centering
            $x_{0_{4}}=[5,-7]\Longrightarrow$  Avaliação Direta de $f(x)$
            \label{tab:tbld} 
            \writetable{\tbld}
            \bigskip
            \captionof{table}{Resultados relacionados ao esforço computacional e precisão considerando o ponto inicial $x_{0_{4}}$}
        \end{minipage}

        \subsubsection{Conclusão Parcial}
            Para funções quadráticas, os métodos Quase-Newton:
            \begin{itemize}
            \item {Tempo (s):} As aproximações Quadráticas para $F(s)$ apresentam um tempo médio de execução 10 vezes maior. O quadrante do ponto inicial não apresenta alteração no resultado. A média dos tempos é semelhante para todos os métodos.
            \item {Iterações:} Não variou significativamente para qualquer uma das condições.
            \item {Avaliações de$F(x)$:} Aumentam consideravelmente quando se emprega a técnica da avaliação direta da função para o cálculo da seção áurea. 
            \item {Erro $X_{sol}$ (\%):} O erro é muito pequeno para todas as todas as variações, $\leq 10^{-9}$. Aparentemente o quadrante do ponto inicial aumenta em $10^5$ o tempo erro de $X_{sol}$. Pontos iniciais nos $1\degree$ e $4\degree$ quadrantes tem erros maiores. Porém ainda são virtualmente zero.
            \item {Erro $F(x_{sol})$ (\%):} \textbf{Sempre} zeram o erro da Função objetivo calculado no ponto da solução para todas as variações. 
            \end{itemize}
            
    \subsection{Segunda Função Analisada}

    A Segunda função analisada consiste na função apresentada na seção \ref{sec:secfun}.

    A função objetivo apresentada na equação (\ref{eq:secfuntst}) possui duas variáveis de decisão. Nesse caso, o número máximo de iterações, calculado por meio da equação \ref{eq:maxevals}, é igual a 400 e os limites das variáveis de decisão $[\ -10 , 10 ]\ $. No intuito de obter uma função quadrática, o parâmetro $a$ é igual a zero, resultando em:

    \begin{equation*} 
        f(x) = 12*x_1^2 - 4*x_2^2 - 12*x_1x_2 + 2*x_1
    \end{equation*}
    
    Na execução dos algoritmos que implementam os métodos Quase-Newton deverão ser considerados dois pontos iniciais definidos aleatoriamente pelo aluno. Em um desses dois pontos iniciais será usada a técnica da seção áurea da avaliação direta da função e, no outro, por meio das aproximações quadráticas para a função a cada iteração. 

        \begin{minipage}{\linewidth}
            \centering
            $x_0=[x_1,x_2]\Longrightarrow$  Avaliação Direta de $f(x)$
            \label{tab:tble} 
            \writetable{\tble}
            \bigskip
            \captionof{table}{Resultados relacionados ao esforço computacional e precisão considerando o ponto inicial $x_{0_{1}}$}
        \end{minipage}
        
        \begin{minipage}{\linewidth}
            \centering
            $x_0=[x_1,x_2]\Longrightarrow$  Aproximações Quadráticas para $f(x)$
            \label{tab:tblf} 
            \writetable{\tblf}
            \bigskip
            \captionof{table}{Resultados relacionados ao esforço computacional e precisão considerando o ponto inicial $x_{0_{2}}$}
        \end{minipage}

         \subsubsection{Conclusão Parcial}
            Para funções quadráticas, os métodos Quase-Newton:
            \begin{itemize}
            \item {Tempo (s):} 
            \item {Iterações:} 
            \item {Avaliações de$F(x)$:} 
            \item {Erro $X_{sol}$ (\%):} 
            \item {Erro $F(x_{sol})$ (\%):} 
            \end{itemize}        
   
    \subsection{Terceira Função Analisada}

    A terceira função analisada consiste na função da seção \ref{sec:terfun}.
    
    A função objetivo apresentada acima possui duas variáveis de decisão. Nesse caso, o número máximo de iterações, calculado por meio da equação \ref{eq:maxevals}, é igual a 400 e os limites das variáveis de decisão definem o seguinte intervalo $[\ 0 , 3 ]\ $. Nesse intervalo, embora a função $f(x)$ possua dois termos cúbicos, a mesma se comporta como uma função quadrática. Sabendo que $a=\sqrt{2}$, tem-se que:

    \begin{equation*} 
        f(x) = -8*x_1*x_2+2*x_2*x_1^3+2*x_1*x_2^3
    \end{equation*}

    Na execução dos algoritmos que implementam os métodos Quase-Newton, na análise dessa função, serão considerados dois pontos iniciais definidos aleatoriamente pelo aluno. Em um desses dois pontos iniciais será utilizada a técnica da seção áurea feita através da avaliação direta da função e, no outro, por meio das aproximações quadráticas para a função a cada iteração.

        \begin{minipage}{\linewidth}
            \centering
            $x_0=[x_1,x_2]\Longrightarrow$  Avaliação Direta de $f(x)$
            \label{tab:tblg} 
            \writetable{\tblg}
            \bigskip
            \captionof{table}{Resultados relacionados ao esforço computacional e precisão considerando o ponto inicial $x_{0_{1}}$}
        \end{minipage}
        
        \begin{minipage}{\linewidth}
            \centering
            $x_0=[x_1,x_2]\Longrightarrow$  Aproximações Quadráticas para $f(x)$
            \label{tab:tblh} 
            \writetable{\tblh}
            \bigskip
            \captionof{table}{Resultados relacionados ao esforço computacional e precisão considerando o ponto inicial $x_{0_{2}}$}
        \end{minipage}

        \subsubsection{Conclusão Parcial}
            Para funções quadráticas, os métodos Quase-Newton:
            \begin{itemize}
            \item {Tempo (s):} 
            \item {Iterações:} 
            \item {Avaliações de$F(x)$:} 
            \item {Erro $X_{sol}$ (\%):} 
            \item {Erro $F(x_{sol})$ (\%):} 
            \end{itemize}
\newpage