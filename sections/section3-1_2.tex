% Experimentos e Resultados

\pgfplotstableread{data/sect3s1/sec3s1tbl6.txt}{\tble}
\pgfplotstableread{data/sect3s1/sec3s1tbl7.txt}{\tblf}

\pgfplotstableread{data/sect3s1/sect3s1f2_TEnd_Medio.txt}{\tblanovaT}
\pgfplotstableread{data/sect3s1/sect3s1f2_N_Iter_Medio.txt}{\tblanovaNI}
\pgfplotstableread{data/sect3s1/sect3s1f2_N_Evals_F_Medio.txt}{\tblanovaNE}
\pgfplotstableread{data/sect3s1/sect3s1f2_Erro_Xsol_P_Medio.txt}{\tblanovaEX}
\pgfplotstableread{data/sect3s1/sect3s1f2_Erro_F_P_Medio.txt}{\tblanovaEF}


A Segunda função analisada consiste na função apresentada na seção \ref{sec:secfun}.

A função objetivo apresentada na equação (\ref{eq:secfuntst}) possui duas variáveis de decisão. Nesse caso, o número máximo de iterações, calculado por meio da equação \ref{eq:maxevals}, é igual a 400 e os limites das variáveis de decisão $[\ -10 , 10 ]\ $. No intuito de obter uma função quadrática, o parâmetro $a$ é igual a zero, resultando em:

\begin{equation*} 
    f(x) = 12*x_1^2 - 4*x_2^2 - 12*x_1x_2 + 2*x_1
\end{equation*}

Na execução dos algoritmos que implementam os métodos Quase-Newton deverão ser considerados dois pontos iniciais definidos aleatoriamente pelo aluno. Em um desses dois pontos iniciais será usada a técnica da seção áurea da avaliação direta da função e, no outro, por meio das aproximações quadráticas para a função a cada iteração. 

\begin{minipage}{\linewidth}
    \centering
    $x_0=[x_1,x_2]\Longrightarrow$  Avaliação Direta de $f(x)$
    \label{tab:tble} 
    \writetable{\tble}
    \bigskip
    \captionof{table}{Resultados relacionados ao esforço computacional e precisão considerando o ponto inicial $x_{0_{1}}$}
\end{minipage}

\begin{minipage}{\linewidth}
    \centering
    $x_0=[x_1,x_2]\Longrightarrow$  Aproximações Quadráticas para $f(x)$
    \label{tab:tblf} 
    \writetable{\tblf}
    \bigskip
    \captionof{table}{Resultados relacionados ao esforço computacional e precisão considerando o ponto inicial $x_{0_{2}}$}
\end{minipage}  
   
\subsubsection{Análise Anova}

    \begin{minipage}[h!]{\linewidth}
        \centering
        \hrule
        \vspace{2mm}
        {Tabela Anova de 2 Fatores com blocos Aleatorizados}
        \vspace{2mm}
        \noindent
        \hrule 
        \vspace{2mm}
        Tempo de Processamento Médio\\
        \label{tab:tblDa} 
        \writeanova{\tblanovaT}\par
        \bigskip
        \centering
        Número de Iterações\\
        \label{tab:tblDb} 
        \writeanova{\tblanovaNI}\par
        \bigskip
        \centering
        Número de Avaliações de $f(x)$\\
        \label{tab:tblDc} 
        \writeanova{\tblanovaNE}\par
        \bigskip
        \centering
        {Erro de $X_{sol}(\%)$}\\
        \label{tab:tblDb} 
        \writeanova{\tblanovaEX}\par
        \bigskip
        \centering
        {Erro de $F(X_{sol})(\%)$}\\
        \label{tab:tblDb} 
        \writeanova{\tblanovaEF}\par
        \vspace{2mm}
        \hrule
        \vspace{2mm}
        \captionof{table}{Análise de variância dos Métodos Quase-Newton e da técnica da seção áurea para a segunda função analisada do primeiro experimento}
    \end{minipage}

    Para funções quadráticas, os métodos Quase-Newton:
    \begin{itemize}
        \item {Tempo (s):} O teste F da tabela anova indica que, tanto o método de cálculo da Seção Áurea quanto o método de otimização, interferem significativamente para a variação dos tempos médios de execução. Sendo a influência do método de cálculo da Seção Áurea é mais significativa. A observação dos resultados aponta que as aproximações Quadráticas para $F(s)$ apresentam um tempo médio de execução 10 vezes menor. Enquanto isso, o método otimização interfere com uma variância no tempo de execução significativamente menor, quase desprezível se comparada a variação causada pelo método de cálculo da Seção Áurea. 
        \item {Iterações:} O teste F da tabela anova considera que nenhum dos fatores é capaz de causar alteração significativa do número de iterações do algoritmo, devido a esta variável ter valor constante e igual a 6 para todos os blocos executados.
        \item {Avaliações de $F(x)$:} O teste F da tabela anova indica que o método de cálculo da Seção Áurea interfere significativamente para a variação do Número de avaliações de $F(x)$. A observação dos resultados aponta que as aproximações Quadráticas para $F(s)$ apresentam um tempo médio de execução 50 vezes menor. 
        \item {Erro $X_{sol}$ (\%):} O teste F da tabela anova indica que o Método de otimização interfere significativamente para a variação do Erro $X_{sol}$ (\%). Pela observação dos dados pode-se constatar que a combinação entre o uso do médoto de Otimização DFP provê o menor valor para o Erro $X_{sol}$ com uma redução da ordem de 10 vezes o valor das demais combinações.
        \item {Erro $F(x_{sol})$ (\%):} \textbf{Sempre} zeram o erro da Função objetivo calculado no ponto da solução para todas variações de todos os fatores. O teste F da tabela anova considera que nenhum dos fatores é capaz de causar alteração significativa do Erro $F(x_{sol})$ (\%), devido a esta variável ter valor constante e igual a 0 para todos os blocos executados.
    \end{itemize}   